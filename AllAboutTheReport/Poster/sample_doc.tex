\documentclass[]{beamer}


\usepackage[scale=0.85, size=custom, width=84, height=64]{beamerposter} 
\usetheme{vegaposter} 


\addbibresource{refs.bib}
\usepackage{lipsum}

\title{Optimal execution problem in Obizhaeva--Wang framework}
\author{Vsevolod Zaostrovsky, Peter Shkenev}
\supervisor{Anton O. Belyakov, Alexey Savin}
\researchgroup{Market Microstructure}

\begin{document}
\nocite{*} % This is needed to make sure that all references are included in the bibliography

\begin{frame}[t]
    \begin{columns}[t] % The whole poster consists of three major columns, the second of which is split into two columns twice - the [t] option aligns each column's content to the top
     
    \begin{column}{\lrmargin}\end{column} % Empty spacer column
    
    \begin{column}{\onecolwid} % The first column
     
    %----------------------------------------------------------------------------------------
    %	INTRODUCTION
    %----------------------------------------------------------------------------------------
    
    \begin{block}{Introduction}
    
        Issues related to the structure of the order book are very important for the industry, 
        so in recent decades a new young and interesting science has been built around these issues. In our research, we are looking for a way 
        to connect the latest advances in this science associated with various variations of the Obizhaeva--Wang model with the needs of industry.
    
    \end{block}

    \begin{block}{Optimal execution problem}
    
        The idea of that problem is quite simple. If one wants to sell or buy an amount of an asset large enough to have a significant 
        impact on the market, he, obviously should not do it by one order: it would be very expensive, since a large order 
        would remove all the upper levels in the limit order book. Therefore, in practice, all large orders are split into a large number of small ones. 
        For example, one can simply divide an order into N equal parts and sell them at regular intervals (this is called TWAP). 
        To find a better solution, we consider the OW model, in which terms the problem has the following form: \par  


    % \begin{block}{Why the OW model?}
    %     The supply/demand of financial securities is in general not perfectly elastic. This fact is true even for liquid European markets, if we talk about 
    %     much less liquid Russian markets, neglecting this fact can be disastrous. The main difference between Obizhaeva's model and others is precisely 
    %     that resiliency --- the speed at which supply/demand recovers to its steady state after a trade --- plays a key role in it.
    % \end{block}

        
        \begin{align*}
           J_0 &= \min _{\{x_0 \cdots x_N \}} E_0 \left[ \sum _{n=0}^N [A_{t_n} + x_n /(2q)] x_n\right],  \\
           A_{t_n} &= F_{t_n} + \lambda (X_0 - X_{t_n}) + s/2 + \sum _{i=0}^{n-1} x_i \kappa e^{- \rho \tau (n - i)}.
        \end{align*}
        
       Here:
       \begin{itemize}
        \item The trader has to buy $\mathbf{X_0}$ units of a security over a fixed time period $[0,T]$. $x_{t_n}$ 
        --- the trade size at $t_n = \tau n$, where $\tau = T / N$. $X_{t_n} := X_0 - \sum _{t_k < t_n} x_{t_k}$. 
        \item $B_{t_n}$ and $A_{t_n}$ --- bid and ask prices at $t_n$. $V_{t_n} = \frac{A_{t_n} + B_{t_n}}{2}$ 
        --- the mid-quote price; $s$ --- the bid–ask spread.
        \item $F_t$ --- the fundamental value of the security.
        % \item $q(P)$ --- the density of limit orders to sell at price $P$.
        \item Parameter $\lambda$ captures the permanent price impact.
        \item Parameter $q$ depends on LOB density. 
        \item $\kappa = \frac{1}{q} - \lambda $
        \item Parameter $\rho$ captures the resiliency.

       \end{itemize}
        \end{block}
    
    
    \end{column} 
    \begin{column}{\sepwid}\end{column} % Empty spacer column
    
    \begin{column}{\onecolwid} % Begin a column which is two columns wide (column 2)
    

    \begin{block}{Optimal execution strategy }
    
    Proposition 2 from \cite{obizhaeva2013optimal} gives an optimal strategy for big $N$.
    
    \begin{theorem}
        As $N \rightarrow \infty$, the optimal execution strategy becomes:
        \begin{align*}
            & \lim _{N \rightarrow \infty} x_0 = x_{t = 0} = \frac{X_0}{\rho T + 2}, \\
            & \lim _{N \rightarrow \infty} x_n / (T/N) = \dot X _t = \frac{\rho X_0}{\rho T + 2}, \;\;\;\;\;\; t \in (0, T), \\
            & \lim _{N \rightarrow \infty} x_n / (T/N) = x_{t=T}=  \frac{X_0}{\rho T + 2},  \\
        \end{align*}
        where $x_0$ is the trade at the beginning of trading period, $x_N$ is the trade at the end of trading
        period, and $\dot X _t$ is the speed of trading in between these trades.
    \end{theorem}
    The key question here is:
    \end{block}

    \begin{block}{How to find $\rho$?}
        
        We provide our methodology to find $\rho$. We find it, considering time series on elements of the model 
        that can be calculated from market data. As an example, we are going to consider the regression:
        \begin{theorem}
            In regression:
            \begin{equation*}
                \frac{\Delta A_{k+2}}{\Delta t_{k+2}} - \frac{\Delta A_{k+1}}{\Delta t_{k+1}} 
        = - \rho \Delta A_{k+1} + \rho \lambda x_{k+1} + (\alpha + \lambda) (\frac{x_{k+2}}{\Delta t_{k+2}} - \frac{x_{k+1}}{\Delta t_{k+1}}).
            \end{equation*}
            $\rho$ the same as in OW model.
        \end{theorem}
        \begin{proof}
            \[ D_{k+1} - D_k = -\rho D_k \Delta t_{k+1} + \alpha x_{k+1} \]
            \[ \Delta t_{k+1} := t_{k+1} - t_k, \; \; \; \; \; D_k := D_{t_k}, \; \; \; \; \; x_{k}:= x_{t_k}, \; \; \; \; \; \Delta D_{k+1} := D_{k+1} - D_k . \]
            \[ V_{k+1} - V_k = \lambda x_{k+1} \rightarrow \Delta D_{k+1} = \Delta A_{k+1} - \lambda x_k \] 
            \[ \frac{\Delta D_{k+1}}{\Delta t_{k+1}} = - \rho D_k + \alpha \frac{x_{k+1}}{\Delta t_{k+1}} \] 
            \[ \frac{\Delta D_{k+2}}{\Delta t_{k+2}} - \frac{\Delta D_{k+1}}{\Delta t_{k+1}} 
                = - \rho \Delta D_{k+1} + \alpha ( \frac{x_{k+2}}{\Delta t_{k+2}} - \frac{x_{k+1}}{\Delta t_{k+1}}) \] 
        \end{proof}
    \end{block}

   
    
    %----------------------------------------------------------------------------------------
    \end{column}
    
    
    \begin{column}{\sepwid}\end{column} % Empty spacer column
    
    \begin{column}{\onecolwid} % The third column
    
        %----------------------------------------------------------------------------------------
        %	CONCLUSION
        %----------------------------------------------------------------------------------------
        
        \begin{block}{Purposes}
        \begin{itemize}
            \item Propose methodology for fitting OWM factors and use it to get optimal execution strategy.
            \item Propose a backtest procedure for the optimal execution algorithm, implement it and compare the algorithm with TWAP.
        \end{itemize}
        
        \end{block}
        
        %----------------------------------------------------------------------------------------
        %	REFERENCES
        %----------------------------------------------------------------------------------------
        
        \begin{block}{References}
        
    %    \nocite{*} % Insert publications even if they are not cited in the poster
        \printbibliography \vspace{0.75in}
        
        \end{block}
        
        \end{column} % End of the third column
    
    \begin{column}{\lrmargin}\end{column} % Empty spacer column
    
    \end{columns} % End of all the columns in the poster
    \end{frame} % End of the enclosing frame
\end{document}