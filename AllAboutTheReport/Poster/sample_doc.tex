\documentclass[]{beamer}


\usepackage[scale=0.8, size=custom, width=84, height=64]{beamerposter} 
\usetheme{vegaposter} 


\addbibresource{refs.bib}
\usepackage{lipsum}

\title{Optimal execution problem in Obizhaeva--Wang framework}
\author{Vsevolod Zaostrovsky, Peter Shkenev}
\supervisor{Anton Belyakov, Alexey Savin}
\researchgroup{Market Microstructure}

\begin{document}
\nocite{*} % This is needed to make sure that all references are included in the bibliography

\begin{frame}[t]
    \begin{columns}[t] % The whole poster consists of three major columns, the second of which is split into two columns twice - the [t] option aligns each column's content to the top
     
    \begin{column}{\lrmargin}\end{column} % Empty spacer column
    
    \begin{column}{\onecolwid} % The first column
     
    %----------------------------------------------------------------------------------------
    %	INTRODUCTION
    %----------------------------------------------------------------------------------------
    
    \begin{block}{Introduction}
    
        The introduction of resiliency --- the speed at which supply/demand recovers to its steady state after 
        a trade --- characterizes the beginning of a new stage in the development of optimal execution models.
        In our research we develope a practical way to utilize that object. \par
        The supply/demand of financial securities is in general not perfectly elastic. This fact is true even for liquid European markets, if we talk about 
        much less liquid Russian markets, neglecting this fact can be disastrous. The main difference between OW model and others is precisely 
        that resiliency plays a key role in it.
        
    
    \end{block}

    \begin{block}{Optimal execution problem}
    
        If one wants to sell or buy an amount of an asset large enough to have a significant 
        impact on the market, he, obviously, should not do it by one order: it would be very expensive, since a large order 
        would remove all the upper levels in the limit order book. Therefore, in practice, all large orders are split into a large number of small ones. 
        For example, one can simply divide an order into N equal parts and sell them at regular intervals (this is called TWAP). 
        To find a better solution, we consider the OW model, in which terms the problem has the following form: \par  


        
        \begin{align*} \label{oEproblem}
           J_0 &= \min _{\{x_0 \cdots x_N \}} E_0 \left[ \sum _{n=0}^N [A_{t_n} + x_n /(2q)] x_n\right],  \\
           A_{t_n} &= F_{t_n} + \lambda (X_0 - X_{t_n}) + s/2 + \sum _{i=0}^{n-1} x_i \kappa e^{- \rho \tau (n - i)}.
        \end{align*}
        
       Here:
       \begin{itemize}
        \item The trader has to buy $\mathbf{X_0}$ units of a security over a fixed time period $[0,T]$. 
        \item $x_{t_n}$ 
        --- the trade size at $t_n = \tau n$, where $\tau = T / N$. 
        \item $X_{t_n} := X_0 - \sum _{t_k < t_n} x_{t_k}$. 
        \item $B_{t_n}$ and $A_{t_n}$ --- bid and ask prices at $t_n$. 
        \item $V_{t_n} = \frac{A_{t_n} + B_{t_n}}{2}$ 
        --- the mid-quote price; 
        \item $s$ --- the bid–ask spread.
        \item $F_t$ --- the fundamental price of the security.
        \item $q(P)$ --- the density of limit orders to sell at price $P$.
        % \item $q$, $\lambda$ and $\rho$ is a LOB density, the permanent price impact and the resiliency.
        \item Parameter $\lambda$ captures the permanent price impact.
        \item Parameter $q$ is a LOB density. 
         \item $\kappa = \frac{1}{q} - \lambda $
        \item Parameter $\rho$ captures the resiliency.

       \end{itemize}
        \end{block}

    
    
    \end{column} 
    \begin{column}{\sepwid}\end{column} % Empty spacer column
    
    \begin{column}{\onecolwid} % Begin a column which is two columns wide (column 2)

        \begin{block}{The key question: how to find $\rho$?}
        
            We provide our methodology to find $\rho$. We find it, considering time series on elements of the model 
            that can be calculated from market data. As an example, we are going to consider the regression:
            \begin{alertblock}{Our method to find $\rho$}
                \begin{equation*}
                    \frac{\Delta A_{k+2}}{\Delta t_{k+2}} - \frac{\Delta A_{k+1}}{\Delta t_{k+1}} 
            = - \rho \Delta A_{k+1} + \rho \lambda x_{k+1} + (\kappa + \lambda) (\frac{x_{k+2}}{\Delta t_{k+2}} - \frac{x_{k+1}}{\Delta t_{k+1}}).
                \end{equation*}
            \end{alertblock}
            Where $\Delta A_{k+2}$ is an ask change after execution of the limit order with the depth $x_k$ 
            and $\Delta t_{k+2}$ is a time time between two adjacent orders of dataset.

        \end{block}

        \begin{block}{Optimal execution strategies }
            Using the fitted parameter $\rho$, now we are able to execute the limit of the following strategy.
            To fit the discrete variant one needs more parameters and more regression. 

            \begin{alertblock}{Discrete optimal execution strategy}
                The solution to the optimal execution problem is
                \begin{multline*}
                    x_n = - \frac{1}{2} \delta_{n + 1} [D_{t_n} (1 - \beta_{n + 1} e^{ - \rho \tau} + 2 \kappa \gamma_{n+1} e^{ - 2 \rho \tau}) 
                    - \\ - X_{t_n} (\lambda + 2 \alpha_{n+1} - \beta_{n+1}\kappa e^{ - \rho \tau}) ], 
                \end{multline*}
                with $x_N = X_N$ and $D_t = A_t - V_t - s/2$. The expected cost for future trades under the optimal
                strategy is determined according to
                \begin{equation*}
                    J_{t_n} = (F_{t_n} + s/2) X_{t_n} + \lambda X_0 X_{t_n} + \alpha_n X_{t_n} ^2 + \beta_{n} D_{t_n} X_{t_n} + \gamma_n D_{t_n}^2, 
                \end{equation*}
                where the coefficients $\alpha_{n+1}$, $\beta_{n+1}$, $\gamma_{n+1}$, and $\delta_{n+1}$ are determined recursively as follows:
                \begin{equation*}
                    \alpha_{n} = \alpha_{n+1} - \frac{1}{4} \delta _{n+1} (\lambda + 2 \alpha_{n+1} - \beta_{n+1} \kappa e^{- \rho \tau})^2, 
                \end{equation*}
                \begin{multline*}
                    \beta_{n} =  \beta_{n+1} e^{- \rho \tau}  + 
                    \\ + \frac{1}{2} \delta _{n+1} (1 - \beta_{n+1} e^{- \rho \tau} 
                     + 2 \kappa \gamma_{n+1} e^{- 2 \rho \tau}) (\lambda + 2 \alpha_{n+1} - \beta_{n+1} \kappa e^{-\rho \tau}), 
                \end{multline*}
                \begin{equation*}
                     \gamma_n =   \gamma_{n+1} e^{- 2 \rho \tau} - \frac{1}{4} \delta _{n+1} (1 - \beta _{n+1} e^{- \rho \tau} 
                + 2 \gamma _{n+1} \kappa e^{- 2 \rho \tau})^2, 
                \end{equation*}
                with $\delta_{n+1} = [1/(2q) + \alpha_{n+1} - \beta_{n+1} \kappa e^{-\rho \tau} + \gamma _{n+1} \kappa ^2 e^{- 2 \rho \tau}]^{-1}$ and terminal conditions
                \begin{equation*}
                    \alpha_{N} = 1/(2q) - \lambda, \;\;\;\;\;\;\; \beta_N = 1, \;\;\;\;\;\;\; \gamma_N = 0.
                \end{equation*}
            \end{alertblock}
    
            % Proposition 2 from \cite{obizhaeva2013optimal} gives an optimal strategy for big $N$.
            
            \end{block}
    
   
    
    %----------------------------------------------------------------------------------------
    \end{column}
    
    
    \begin{column}{\sepwid}\end{column} % Empty spacer column
    
    \begin{column}{\onecolwid} % The third column
    
        %----------------------------------------------------------------------------------------
        %	CONCLUSION
        %----------------------------------------------------------------------------------------
        
        \begin{alertblock}{Limit of the discrete optimal execution strategy}
            As $N \rightarrow \infty$, the optimal execution strategy becomes:
            \begin{align*}
                % & \lim _{N \rightarrow \infty} x_0 = x_{t = 0} = \frac{X_0}{\rho T + 2}, \\
                & \lim _{N \rightarrow \infty} x_n / (T/N) = \dot X _t = \frac{\rho X_0}{\rho T + 2}, \;\;\;\;\;\; t \in (0, T), \\
                & \lim _{N \rightarrow \infty} x_0 = x_{t = 0} = \lim _{N \rightarrow \infty} x_n / (T/N) = x_{t=T}=  \frac{X_0}{\rho T + 2}.  %\\
            \end{align*}
            % where $x_0$ is the trade at the beginning of trading period, $x_N$ is the trade at the end of trading
            % period, and $\dot X _t$ is the speed of trading in between these trades.
        \end{alertblock}

        \begin{block}{Problems}
            \begin{itemize}
                \item The task formulated in the KPI is not directly related to the article \cite{obizhaeva2013optimal}. 
                \cite{obizhaeva2013optimal} and \cite{velu2020algorithmic} pose the problem significantly differently. 
                Similar terminology we have found in \cite{webster2023handbook}, but 
                we did not find the theory to work with in that framework.
                \item The data we previously had did not have a sufficient level of detail to extract accurate model values. 
                It was necessary to make assumptions that significantly distorted the final result. 
                New data will require significant time to parse and research. Anyway, data work is very complicated.
                \item This area is very rich and complicated. It is very hard to do even easy steps in the theoretical research, 
                because we did not have courses on that theory. 
                
            \end{itemize}

            
            \end{block}


        \begin{block}{Purposes}
        \begin{itemize}
            \item Develope methodology for fitting OWM factors and use it to get optimal execution strategy.
            \item Compare different approaches of measuring resiliency on l3 data.
            \item Compare discrete and limit OW execution strategy.
            \item Propose a backtest procedure for the optimal execution algorithm, implement it and compare the algorithm with TWAP
            on real market data.
        \end{itemize}
        
        \end{block}
        
        %----------------------------------------------------------------------------------------
        %	REFERENCES
        %----------------------------------------------------------------------------------------
        
        \begin{block}{References}
        
    %    \nocite{*} % Insert publications even if they are not cited in the poster
        \printbibliography \vspace{0.75in}
        
        \end{block}
        
        \end{column} % End of the third column
    
    \begin{column}{\lrmargin}\end{column} % Empty spacer column
    
    \end{columns} % End of all the columns in the poster
    \end{frame} % End of the enclosing frame
\end{document}