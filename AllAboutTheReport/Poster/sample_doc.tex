\documentclass[]{beamer}


\usepackage[scale=0.85, size=custom, width=84, height=64]{beamerposter} 
\usetheme{vegaposter} 


\addbibresource{refs.bib}
\usepackage{lipsum}

\title{Optimal execution problem in Obizhaeva--Wang framework}
\author{Vsevolod Zaostrovsky, Peter Shkenev}
\supervisor{Anton O. Belyakov, Alexey Savin}
\researchgroup{Market Microstructure}

\begin{document}
\nocite{*} % This is needed to make sure that all references are included in the bibliography

\begin{frame}[t]
    \begin{columns}[t] % The whole poster consists of three major columns, the second of which is split into two columns twice - the [t] option aligns each column's content to the top
     
    \begin{column}{\lrmargin}\end{column} % Empty spacer column
    
    \begin{column}{\onecolwid} % The first column
     
    %----------------------------------------------------------------------------------------
    %	INTRODUCTION
    %----------------------------------------------------------------------------------------
    
    \begin{block}{Introduction}
    
        Issues related to the structure of the order book are very important for the industry, 
        so in recent decades a new and interesting science has been built around these issues. In our research, we are looking for a way 
        to connect the latest advances in this science associated with various variations of the Obizhaeva--Wang model with the needs of industry.
    
    \end{block}

    \begin{block}{Optimal execution problem}
    
        The idea of that problem is quite simple. If one wants to sell or buy an amount of an asset large enough to have a significant 
        impact on the market, he, obviously, should not do it by one order: it would be very expensive, since a large order 
        would remove all the upper levels in the limit order book. Therefore, in practice, all large orders are split into a large number of small ones. 
        For example, one can simply divide an order into N equal parts and sell them at regular intervals (this is called TWAP). 
        To find a better solution, we consider the OW model, in which terms the problem has the following form: \par  


        
        \begin{align*}
           J_0 &= \min _{\{x_0 \cdots x_N \}} E_0 \left[ \sum _{n=0}^N [A_{t_n} + x_n /(2q)] x_n\right],  \\
           A_{t_n} &= F_{t_n} + \lambda (X_0 - X_{t_n}) + s/2 + \sum _{i=0}^{n-1} x_i \kappa e^{- \rho \tau (n - i)}.
        \end{align*}
        
       Here:
       \begin{itemize}
        \item The trader has to buy $\mathbf{X_0}$ units of a security over a fixed time period $[0,T]$. 
        \item $x_{t_n}$ 
        --- the trade size at $t_n = \tau n$, where $\tau = T / N$. 
        \item $X_{t_n} := X_0 - \sum _{t_k < t_n} x_{t_k}$. 
        \item $B_{t_n}$ and $A_{t_n}$ --- bid and ask prices at $t_n$. 
        \item $V_{t_n} = \frac{A_{t_n} + B_{t_n}}{2}$ 
        --- the mid-quote price; 
        \item $s$ --- the bid–ask spread.
        \item $F_t$ --- the fundamental value of the security.
        % \item $q(P)$ --- the density of limit orders to sell at price $P$.
        \item Parameter $\lambda$ captures the permanent price impact.
        \item Parameter $q$ depends on LOB density. 
        \item $\kappa = \frac{1}{q} - \lambda $
        \item Parameter $\rho$ captures the resiliency.

       \end{itemize}
        \end{block}
    
    
    \end{column} 
    \begin{column}{\sepwid}\end{column} % Empty spacer column
    
    \begin{column}{\onecolwid} % Begin a column which is two columns wide (column 2)

    \begin{block}{Why the OW model?}
        The supply/demand of financial securities is in general not perfectly elastic. This fact is true even for liquid European markets, if we talk about 
        much less liquid Russian markets, neglecting this fact can be disastrous. The main difference between Obizhaeva's model and others is precisely 
        that resiliency --- the speed at which supply/demand recovers to its steady state after a trade --- plays a key role in it.
    \end{block}
    

    \begin{block}{Optimal execution strategy }
    
    % Proposition 2 from \cite{obizhaeva2013optimal} gives an optimal strategy for big $N$.
    
    \begin{alertblock}{Optimal execution strategy in OW framework}
        As $N \rightarrow \infty$, the optimal execution strategy becomes:
        \begin{align*}
            & \lim _{N \rightarrow \infty} x_0 = x_{t = 0} = \frac{X_0}{\rho T + 2}, \\
            & \lim _{N \rightarrow \infty} x_n / (T/N) = \dot X _t = \frac{\rho X_0}{\rho T + 2}, \;\;\;\;\;\; t \in (0, T), \\
            & \lim _{N \rightarrow \infty} x_n / (T/N) = x_{t=T}=  \frac{X_0}{\rho T + 2},  \\
        \end{align*}
        where $x_0$ is the trade at the beginning of trading period, $x_N$ is the trade at the end of trading
        period, and $\dot X _t$ is the speed of trading in between these trades.
    \end{alertblock}
    The key question here is:
    \end{block}

    \begin{block}{How to find $\rho$?}
        
        We provide our methodology to find $\rho$. We find it, considering time series on elements of the model 
        that can be calculated from market data. As an example, we are going to consider the regression:
        \begin{alertblock}{Our method to find $\rho$}
            In regression:
            \begin{equation*}
                \frac{\Delta A_{k+2}}{\Delta t_{k+2}} - \frac{\Delta A_{k+1}}{\Delta t_{k+1}} 
        = - \rho \Delta A_{k+1} + \rho \lambda x_{k+1} + (\kappa + \lambda) (\frac{x_{k+2}}{\Delta t_{k+2}} - \frac{x_{k+1}}{\Delta t_{k+1}}).
            \end{equation*}
            $\rho$ the same as in OW model.
        \end{alertblock}
    \end{block}

   
    
    %----------------------------------------------------------------------------------------
    \end{column}
    
    
    \begin{column}{\sepwid}\end{column} % Empty spacer column
    
    \begin{column}{\onecolwid} % The third column
    
        %----------------------------------------------------------------------------------------
        %	CONCLUSION
        %----------------------------------------------------------------------------------------
        
        \begin{block}{Problems}
            \begin{itemize}
                \item It seems that the task formulated in the KPI is more indirectly related to the article \cite{obizhaeva2013optimal}
                than directly. \cite{obizhaeva2013optimal} and \cite{velu2020algorithmic} pose 
                the problem significantly differently. Similar terminology we have found in \cite{webster2023handbook}, but 
                we did not find the theory to work with in that framework, although considerable time was devoted to these vain searches.
                Also, that leaded us to a wrong way and we wasted a lot of time trying to solve the wrong task.
                \item The data we previously had did not have a sufficient level of detail to extract accurate model values. 
                It was necessary to make assumptions and results that significantly distorted the final result. 
                New data will require significant time to parse and research. Anyway, data work is very complicated.
                \item This area is very rich and complicated. It is very hard to do even easy steps in the theoretical research, 
                because we did not have courses on that theory. 
                
            \end{itemize}

            
            \end{block}


        \begin{block}{Purposes}
        \begin{itemize}
            \item Learn to work with new data.
            \item Propose methodology for fitting OWM factors and use it to get optimal execution strategy.
            \item Propose a backtest procedure for the optimal execution algorithm, implement it and compare the algorithm with TWAP.
        \end{itemize}
        
        \end{block}
        
        %----------------------------------------------------------------------------------------
        %	REFERENCES
        %----------------------------------------------------------------------------------------
        
        \begin{block}{References}
        
    %    \nocite{*} % Insert publications even if they are not cited in the poster
        \printbibliography \vspace{0.75in}
        
        \end{block}
        
        \end{column} % End of the third column
    
    \begin{column}{\lrmargin}\end{column} % Empty spacer column
    
    \end{columns} % End of all the columns in the poster
    \end{frame} % End of the enclosing frame
\end{document}