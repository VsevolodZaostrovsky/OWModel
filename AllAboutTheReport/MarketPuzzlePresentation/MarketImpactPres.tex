\documentclass[aspectratio=169]{beamer}

% \renewcommand{\textSupervisors}    {Supervisors}

\usetheme{vega}

\title{Report on the article "The Market Impact Puzzle"}
\subtitle{SRG Market microstructure}
\author{Vsevolod Zaostrovsky}
\institute{Vega Institute Foundation}
\supervisor{Anton O. Belyakov, Anton A. Filatov}
% \date{August 21 -- 28, 2022}

\usepackage[]{lipsum}
\begin{document}
\maketitle

\begin{frame}{Structure}
    \tableofcontents
\end{frame}

\section{The square root model of market impact}
\begin{frame}{The square root model of market impact}
    The square root model of market impact was proposed by Barra (1997), based on empirical
    regularities observed by Loeb (1983), as a practical way for asset managers to measure market
    impact empirically.
    \begin{equation} \label{squarerootmodel}
        G = g(\sigma, P, V; Q) \sim \sigma \left(\frac{|Q|}{V}\right)^{1/2},
    \end{equation}
    here $G$ denotes the percentage cost of executing a
    bet of $Q$ shares of stock with price $P$. \par
    The square root formula is dimensionally consistent: bet size $Q$ is measured in shares, vollume
    $V$ is measured in shares per day, and returns variance $\sigma^2$
    is measured per day, the proportionality coefficient is dimensionless.
\end{frame}

\begin{frame}{Disadvantages of the square root model}
    This square root model is elegantly simple and empirically realistic, but
    has two distinct disadvantages:
    \begin{itemize}
        \item There is still no consensus on whether market impact
              functions can indeed be described exactly by the square root function.
        \item Most theoretical models of market microstructure lead
              to a model of linear market impact, not a square root model. Models with non-linear market impact
              are usually analytically intractable and seem to allow for simple arbitrage strategies.
    \end{itemize}
    \begin{example}
        Given market impact \ref{squarerootmodel}, one could make profits by
        executing over time ten buy trades of 100 shares each and then selling 1000 shares at once.
    \end{example}
\end{frame}

% \begin{frame}{Kyle and Obizhaeva (2016)}
%     Principles of market microstructure invariance:
%     \begin{itemize}
%         \item Dollar risks transferred by bets in business time.
%         \item The dollar costs of executing bets transferring comparable risks
%               are approximately the same across all markets.
%     \end{itemize}
%     These invariance hypotheses imply the empirical conjecture that market impact has the more general functional form:
%     \begin{equation*}
%         G = g(\sigma, P, V; Q) \sim \left( \frac{\sigma^2}{PV}\right) ^{1/3} f\left( \left( \frac{\sigma^2}{PV}\right) ^{1/3} |PQ| \right).
%     \end{equation*}

% \end{frame}

\section{Restrictions Based on Volume and Volatility Equations}

\begin{frame}{Restrictions Based on Volume and Volatility Equations}
    Suppose the percentage market impact G of executing a bet of size $|Q|$ is described by a power function:
    \begin{equation} \label{restrprice}
        G_\beta = \alpha |Q|^\beta.
    \end{equation}
    The parameter $\alpha$ is a dimensional coefficient that may depend on some asset-specific characteristics
    such as volume and volatility, and $\beta$ is the exponent of the market impact function. \par
    Consider the system of three equations:
    \begin{equation} \label{firstsystem}
        \begin{cases}
            V = \gamma E\{|Q|\},        \\
            \sigma^2 = \gamma E\{G^2\}, \\
            E\{G^2\} = \alpha^2 E\{|Q|^{2 \beta}\}.
        \end{cases}
    \end{equation}
    Here $V$ denotes expected trading volume (shares per day), $\sigma^2$ denotes
    expected returns variance, $\gamma$ denotes the expected number of bets (per day).
    
\end{frame}

\begin{frame}{Market Impact Functions}
    The system implies the trading cost:
    \begin{equation} \label{price1}
        G_{\beta} = \frac{\sigma}{\sqrt{\gamma}} \frac{|Q|^\beta}{\sqrt{E\{ |Q|^{2\beta}\}}}
    \end{equation}
    Considering different cases of the function \ref{price1} for different values of the exponent $\beta$
    we can get new models:
    \begin{align*}
        G_0 = \alpha = \frac{\sigma}{\sqrt{\gamma}} = \sqrt{\frac{\sigma^2}{P V} P E\{|Q|\}}.    \\
        G_1 = \alpha |Q| = \sqrt{\frac{\sigma^2}{P V} P E\{|Q|\}} \frac{|Q|}{\sqrt{E\{|Q|^2\}}}. \\
        G_{1/2} = \sigma \sqrt{\left(\frac{|Q|}{V}\right)}.
    \end{align*}
\end{frame}

\section{Restrictions Implied by Transaction Cost Invariance}

\begin{frame}{Restrictions Implied by Transaction Cost Invariance}
    Transaction costs invariance hypothesizes that the ex ante expected dollar cost $E\{|P Q| G\}$ of
    executing a bet, without conditioning on the size of a bet, is constant and equals $C$. From \ref{restrprice}
    we have the fourth equation (besides the system \ref{firstsystem}):
    \begin{equation*}
        C = \alpha P E\{|Q|^{1 + \beta}\}.
    \end{equation*}
    Introducing two dimensionless moment ratios $m$ and $m_\beta$:
    \begin{equation*}
        m := \frac{E\{|Q|\} \sqrt{E\{|Q|^{2 \beta}\}}}{E\{|Q|^{\beta + 1}\}}, \; \; \; \; \; \; \; m_\beta := \frac{(E\{|Q|\})^{\beta + 1}}{E\{|Q|^{\beta + 1}\}};
    \end{equation*}
    one can get:
    \begin{equation*}
        G = \frac{m_\beta C ^{(1- 2 \beta) / 3}}{m ^{2(1 + \beta) / 3}}  \left( \frac{\sigma^2}{PV} \right) ^{(\beta + 1) / 3} |P Q|^\beta .
    \end{equation*}
\end{frame}

\section{Restrictions Implied by Bet Size Invariance}

\begin{frame}{Restrictions Implied by Bet Size Invariance}
    Bet size invariance hypothesizes that the dollar risk a bet
    transfers per unit of business time,
    \begin{equation*}
        I := P Q \frac{\sigma}{\sqrt{\gamma}},
    \end{equation*}
    has an invariant mean $E{|I|}$ for all markets. It can be shown that the transaction cost and bet size invariance hypotheses are closely
    related to each other:
    \begin{equation*}
        C = \frac{1}{m} E\{|P Q|\} \frac{\sigma}{\sqrt{\gamma}} = \frac{1}{m} E\{|I|\}.
    \end{equation*}

\end{frame}

\begin{frame}{A universal market impact formula}
    Let $1/L$ denote the illiquidity measure that is defined as follows:
    \begin{equation*}
        \frac{1}{L} := \frac{C}{E\{|P Q|\}} = \left( \frac{\sigma^2 C}{m^2 P V} \right) ^{1/3}.
    \end{equation*}
    The system \ref{firstsystem} leads to:
    \begin{equation*}
        G = \frac{1}{L} f(Z),
    \end{equation*}
    where $Z$ scales bet size by its mean, in particular:
    \begin{equation*}
        Z := \frac{Q}{E\{|Q|\}} = \frac{P Q}{C L}, \; \; \; \; \; \; \; f(Z) = m_\beta |Z|^\beta .
    \end{equation*}
\end{frame}

\section{A Dimensional Analysis Approach with Leverage Neutrality}

\begin{frame}{A Dimensional Analysis Approach with Leverage Neutrality}
    Leverage Neutrality: the economic costs of trading bundles of risky securities and a cash-equivalent asset are the same
    regardless of any positive or negative amount of cash-equivalent assets included in a bundle.
    \begin{example}
        Suppose an amount of cash equal to aP is added to each share of stock. This decrease in leverage raises the stock price
        to $(1 + a)P$, lowers returns volatility to $\sigma/(1 + a)$, but it does not change $V$, $E\{|Q|\}$, or $\gamma$.
    \end{example}
    Assuming the basis
    $\{V, P, \sigma^2, C\}$ one can apply dimensional analysis, leverage neutrality, and invariance to obtain the market impact formula:
    \begin{equation*}
        G = g(\sigma, P, V; Q) \sim \left( \frac{\sigma^2}{PV}\right) ^{1/3} f\left( \left( \frac{\sigma^2}{PV}\right) ^{1/3} |PQ| \right).
    \end{equation*}

\end{frame}


\section{Conclusions and Future Research}

\begin{frame}{Conclusions}
    \begin{itemize}
        \item Both general approaches in this paper lead to a market impact function of the type:
              \begin{equation*}
                  G = \frac{1}{L} f(Z).
              \end{equation*}
        \item Under both approaches, the square root model is a special knife-edged case of the general
              market impact formula. It is the only case for which the market impact function depends only on volume and volatility.
        \item To satisfy leverage
              neutrality, another market characteristic must be added to the basis $\{V, P, \sigma^2\}$.
    \end{itemize}
\end{frame}

\begin{frame}{Future Research}
    \begin{itemize}
        \item The dimensional analysis approach leaves many unanswered questions.
        \item The author's approach implicitly relies on the assumption that information generates discrete trades of large sizes so that one
              can identify bets and their market impact in the data. How to deal with it?
        \item  Are the variables $C, m,$ and $m_\beta$ approximately constant across markets, countries, and time periods? If
              so, what are their values? Or if not, alternatively, can one identify a set of regimes in which these
              variables are relatively constant? Are there other similar variables that can be almost invariant?
        \item Is there a theory based on financial economics which leads to a square root model
              of market impact?
        \item Is it possible for the author's approach to generate quantitative predictions about the dynamic properties of market impact?

    \end{itemize}
\end{frame}

\end{document}