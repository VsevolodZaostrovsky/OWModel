\begin{appendices} % Do not change this line (if you have appendices). 
                   % Otherwise, completely delete the contents of this file



	\section{Обоснование метода}
        
        \begin{theorem}
                Если $\rho \Delta t << 1$, то в регрессии                                                                                                                                                                                                                                                                                                                                                                                       
                \begin{equation*}
                        \frac{\Delta A_{k+1}}{\Delta t_{k+1}} - \frac{\Delta A_{k}}{\Delta t_{k}}
                        = -\rho \Delta A_k + \rho (\lambda + \kappa) x_{t_k} - \rho \kappa x_{t_{k+1}} + 
                        (\lambda + \kappa) \left(\frac{x_{t_{k+1}}}{\Delta t_{k+1}} - \frac{x_{t_k}}{\Delta t_{k}}\right),
                \end{equation*}
                где $x_{k}$ и $A_{k}$ глубина ордера и аск в момент времени $t_k$, соответственно, \\

                коэффициенты $\rho, \kappa$ и $\lambda$ примерно те же, что в модели Обижаевой--Ванга, описывающей рынок,
                характеризуемый временными рядами: $A_k, \Delta t _k, x_k$.


\end{theorem}
\begin{proof}
        Из определения модели имеем три уравнения:

        \begin{equation} \label{rp1}
                A_{t_k} = V_{t_k} + \frac{s}{2} + \sum _{i=0} ^{k-1} x_{t_i} \kappa e^{- \rho (t_k - t_i)}
        \end{equation}
        \begin{equation}\label{rp2}
                V_{t_{k+1}} = V_{t_k} + \lambda x_{t_k} \rightarrow V_{t_{k+1}} - V_{t_k} = \lambda x_{t_{k}}
        \end{equation}
        \begin{equation} \label{rp3}
                D_{t_k} = A_{t_k} - V_{t_k} - \frac{s}{2}
        \end{equation}

        Из \eqref{rp1} и \eqref{rp3},
        \begin{align*}
                D_{t_k} &= \sum _{i=0} ^{k-1} x_{t_i} \kappa e^{- \rho (t_k - t_i)} \\
                \Delta D_{t_k} &= \sum _{i=0} ^k x_{t_i} \kappa e^{- \rho (t_{k+1} - t_i)} 
                - \sum _{i=0} ^{k - 1} x_{t_i} \kappa e^{- \rho (t_k - t_i)}
                = \sum _{i=0} ^{k - 1} x_{t_i} \kappa (e^{- \rho (t_{k+1} - t_i)} - e^{- \rho (t_k - t_i)})
                + x_{t_k} \kappa e^{- \rho (t_{k+1} - t_k)} = \\
                &= \sum _{i=0} ^{k - 1} x_{t_i} \kappa e^{- \rho (t_k - t_i)} (e^{- \rho (t_{k+1} - t_k)} - 1)
                + x_{t_k} \kappa e^{- \rho (t_{k+1} - t_k)} = D_{t_k} (e^{- \rho \Delta t_k} - 1) + x_{t_k} \kappa e^{- \rho \Delta t_k}.
        \end{align*}
        С другой стороны, из \eqref{rp2} и \eqref{rp3} имеем
        \begin{equation*}
                \Delta D_k = D_{t_{k+1}} - D_{t_k} = A_{t_{k+1}} + V_{t_{k+1}} - A_{t_k} - V_{t_k} = \Delta A_k - \Delta V_k \rightarrow \Delta A_k = \Delta D_k + \Delta V_k .
        \end{equation*}
        Таким образом, получаем уравнение
        \begin{equation}\label{mainregreq}
                \Delta A_k = D_{t_k} (e^{- \rho \Delta t_k} - 1) + x_{t_k} \kappa e^{- \rho \Delta t_k} + \lambda x_{t_k} .
        \end{equation} 
        Таким образом, рассматривая различные интерполяции $e^{- \rho \Delta t_k}$ мы можем получить различные варианты регрессий.
        Например, полагая, что 
        \begin{equation*}
                e^{- \rho \Delta t_k} = 1 - \rho \Delta t_k + O((\rho \Delta t_k) ^2), 
                \text{ где } O((\rho \Delta t_k) ^2) := (\rho \Delta t_k) ^2 \sum_{i=0}^{\infty} \frac{(\rho \Delta t_k)^i}{(i+2)!}
        \end{equation*}
        имеем:
        \begin{equation*}
                \Delta A_{k} 
                % =  D_{t_k} (e^{- \rho \Delta t_k} - 1) + x_{t_k} \kappa e^{- \rho \Delta t_k} + \lambda x_{t_k}
                =  - \rho D_{t_k} \Delta t_k + x_{t_k} \kappa (1 - \rho \Delta t_k) + \lambda x_{t_k} + (x_{t_k} \kappa + D_{t_k}) O((\rho \Delta t_k) ^2).
        \end{equation*}
        Исключим отсюда ряд $D_{t_k}$, поделив на $\Delta t_k$ и рассмотрев разность 
        $A_{t_{k+1}}$ и $A_{t_{k}}$, получим:
        \begin{align*}
                & R := (x_{t_{k+1}} \kappa + D_{t_{k+1}}) O(\rho^2 \Delta t_{k+1}) - (x_{t_k} \kappa + D_{t_k}) O(\rho^2 \Delta t_k) \\
                \frac{\Delta A_{k+1}}{\Delta t_{k+1}} - \frac{\Delta A_{k}}{\Delta t_{k}} &=
                - \rho D_{t_{k+1}} + x_{t_{k+1}} \kappa \left(\frac{1}{\Delta t_{k+1}} - \rho \right) + \lambda \frac{x_{t_{k+1}}}{\Delta t_{k+1}}
                + \rho D_{t_{k}}   - x_{t_{k}}   \kappa \left(\frac{1}{\Delta t_{k}} - \rho \right)   - \lambda \frac{x_{t_k}}    {\Delta t_{k}} 
                + R = \\
                &= -\rho (\Delta A_k - \Delta V_k) + (\lambda + \kappa) \left(\frac{x_{t_{k+1}}}{\Delta t_{k+1}} - \frac{x_{t_k}}{\Delta t_{k}}\right) 
                - \rho \kappa (x_{t_{k+1}} - x_{t_{k}}) + R = \\
                &= -\rho \Delta A_k + \rho (\lambda + \kappa) x_{t_k} - \rho \kappa x_{t_{k+1}} 
                + (\lambda + \kappa) \left(\frac{x_{t_{k+1}}}{\Delta t_{k+1}} - \frac{x_{t_k}}{\Delta t_{k}}\right) + R.
        \end{align*} 
        % Полагая, что $e^{- \rho \Delta t_k} = 1 - \rho \Delta t_k + \frac{1}{2} (\rho \Delta t_k)^2 + o((\rho \Delta t_k) ^3)$, имеем:
        % \begin{align*}
        %         \Delta D_{t_k} &=  D_{t_k} (e^{- \rho \Delta t_k} - 1) + x_{t_k} \kappa e^{- \rho \Delta t_k} 
        %         =  - \rho D_{t_k} \Delta t_k + x_{t_k} \kappa (1 - \rho \Delta t_k) + (x_{t_k} \kappa + D_{t_k}) o((\rho \Delta t_k) ^3) 
        % \end{align*}

        Для произвольной интерполяции экспоненты полиномом первой степени эти рассуждения не проходят:
        \begin{equation*}
                e^{- \rho \Delta t_k} = A - B \Delta t_k + E (\rho \Delta t_k).
        \end{equation*}
        Тогда имеем:
        \begin{align*}
                \Delta A_k &= D_{t_k} (A - B \Delta t_k + R (\rho \Delta t_k) - 1) + 
                x_{t_k} \kappa (A - B \Delta t_k + R (\rho \Delta t_k)) + \lambda x_{t_k} =\\
                &= D_{t_k} (A - 1) - B D_{t_k} \Delta t_k + x_{t_k} \kappa A 
                - B \kappa x_{t_k} \Delta t_k + \lambda x_{t_k} + (x_{t_k} \kappa + D_{t_k}) E (\rho \Delta t_k).
        \end{align*}
        И не ясно, как исключать ряд $D_{t_k}$. Однако, положив $A = 1$, мы всё же можем повторить рассуждения выше и получим возможность
        более точно интерполировать экспоненту. Все выкладки аналогичны таковым для $\rho$:
        \begin{align*}
                & R^1 := E (\rho \Delta t_{k+1}) - E (\rho \Delta t_k) \\
                & \frac{\Delta A_{k+1}}{\Delta t_{k+1}} - \frac{\Delta A_{k}}{\Delta t_{k}} = 
                -B \Delta A_k + B (\lambda + \kappa) x_{t_k} - B \kappa x_{t_{k+1}} 
                + (\lambda + \kappa) \left(\frac{x_{t_{k+1}}}{\Delta t_{k+1}} - \frac{x_{t_k}}{\Delta t_{k}}\right) + R^1.
        \end{align*} 
        Если предположить, что данные действительно подчинены экспоненциальному закону, то, в такой постановке, метод наименьших 
        квадратов фактически будет решать задачу:
        \begin{align*}
        \min _{B \in \mathcal{R}} \max _{x \in [0, t_0]} |e^{-\rho x} - 1 + B x|
        \end{align*} 
        Очевидно, разность под модулем обращается в ноль в двух точках ($0$ и $x_0$), если только прямая не является касательной к экспоненте.
        При этом, функция выпукла в промежутке $[0, x_0]$, а значит имеет там единственную точку экстремума. Из свойств функции ясно, что $B$ 
        является решением задачи в том и только в том случае, когда:
        \begin{equation*}
                - extr \{e^{-\rho x} - 1 + B x \}_{x \in [0, x_0]} = e^{-\rho t_0} - 1 + B t_0.
        \end{equation*} 
        Легко найти точку экстремума $x_*$:
        \begin{equation*}
                \frac{d}{dx} \Big| _{x=x_*} (e^{-\rho x} - 1 + B x) = 0 \rightarrow -\rho e^{-\rho x_*} + B = 0 \rightarrow x_* = -\frac{1}{\rho} \ln \frac{B}{\rho}.
        \end{equation*} 
        Отсюда получаем уравнение, связывающее $\rho$ и $B$:
        \begin{equation*}
                2 - \frac{B}{\rho}\left(1 - \ln \frac{B}{\rho}\right) = e^{- \rho t_0} + B t_0.
        \end{equation*} 


        % Наконец, мы получили искомое уравнение
        % \begin{equation*}
        %         \frac{\Delta A_{k+1}}{\Delta t_{k+1}} - \frac{\Delta A_{k}}{\Delta t_{k}} = 
        %         -\rho \Delta A_k + \rho (\lambda + \kappa) x_{t_k} - \rho \kappa x_{t_{k+1}} + (\lambda + \kappa) \left(\frac{x_{t_{k+1}}}{\Delta t_{k+1}} - \frac{x_{t_k}}{\Delta t_{k}}\right).
        % \end{equation*}

\end{proof}
	
\end{appendices}   % Do not change this line