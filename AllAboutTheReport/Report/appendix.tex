\begin{appendices} % Do not change this line (if you have appendices). 
	% Otherwise, completely delete the contents of this file

	% \begin{theorem}
        %         Модель даёт стратегию оптимального исполнения, аналогичную стратегии оптимального исполнения
        %         оригинальной модели Обижаевой--Ванга, в частности, при $N \rightarrow \infty$:
        %         \begin{align*}
        %             & \lim _{N \rightarrow \infty} x_0 = x_{t = 0} = \frac{X_0}{\hat \rho T + 2}, \\
        %             & \lim _{N \rightarrow \infty} x_n / (T/N) = \dot X _t = \frac{\hat \rho X_0}{\hat \rho T + 2}, \;\;\;\;\;\; t \in (0, T), \\
        %             & \lim _{N \rightarrow \infty} x_0 = x_{t = 0} = \lim _{N \rightarrow \infty} x_n / (T/N) = x_{t=T}=  \frac{X_0}{\hat \rho T + 2}.  %\\
        %         \end{align*}
        %         где $x_0$ первая сделка за отведенный период, $x_N$ --- последняя, и $\dot X _t$ скорость трейдинга между ними.
        %       \end{theorem}
        %       \begin{proof}
        %         Дословно повторяет доказательства A.1 и A.2 из статьи \cite{obizhaeva2013optimal}.
        % \end{proof}

	\section{Что делать, если $\rho$ получается большим?} \label{AppendixBigRho}

	Теорема \ref{lilreg} даёт указание к действию, когда $\rho^2 \Delta t$ мал. Но что делать, если это условие систематически
	нарушается, например, актив настолько ликвиден, что $\rho$ существенно превосходит единицу? \par
	% Если предположить, что данные действительно подчинены экспоненциальному закону, то, в такой постановке, метод наименьших 
	% квадратов фактически будет решать задачу:
	% \begin{equation*}
	%         \sum _{i} \Delta t_i \left(e^{- \rho \Delta t_i} - 1 + B \Delta t_i \right) \rightarrow \min.
	% \end{equation*} 
	% Её решение легко найти аналитически:
	% \begin{equation*}
	%         B = \frac{\sum _{i} \Delta t_i}{\sum _{i} \Delta t_i^2} - \frac{\sum _{i} \Delta t_i e^{-\rho \Delta t_i}}{\sum _{i} \Delta t_i^2} 
	%         = \frac{\sum _{i} \Delta t_i ( 1 - e^{-\rho \Delta t_i})}{\sum _{i} \Delta t_i^2}.
	% \end{equation*} 
	% При $\rho \Delta t_i \rightarrow 0$ имеем $1 - e^{-\rho \Delta t_i} \rightarrow \rho \Delta t_i$, 
	% а значит, $B \rightarrow \rho$, то есть такой подход подтверждает и расширяет полученный ранее вывод. 
	% Однако, эта формула очень неудобна для численного вычисления $\rho(B)$ в иных случаях. Поэтому будем считать,
	% что при вычисленнии коэффициентов решается задача

	\begin{theorem}
		Если считать, что при большом $\rho \Delta t$ регрессия решает задачу
		\begin{equation*}
			\min _{B \in \mathbb{R}} \max _{x \in [0, t_0]} |e^{- \rho x} - 1 + B x|,
		\end{equation*}
		где $t_0$ некоторое "среднее" время между двумя соседними ордерами, то $B$ и $\rho$ связаны уравнением:
		\begin{equation*}
			2 - \frac{B}{\rho}\left(1 - \ln \frac{B}{\rho}\right) = e^{- \rho t_0} + B t_0.
		\end{equation*}
	\end{theorem}
	\begin{proof}
		Очевидно, разность под модулем обращается в ноль в двух точках ($0$ и $x_0$), если только прямая не является касательной к экспоненте.
		При этом, функция выпукла в промежутке $[0, x_0]$, а значит имеет там единственную точку экстремума. Из свойств функции ясно, что $B$
		является решением задачи в том и только в том случае, когда:
		\begin{equation*}
			- extr \{e^{-\rho x} - 1 + B x \}_{x \in [0, x_0]} = e^{-\rho t_0} - 1 + B t_0.
		\end{equation*}
		Легко найти точку экстремума $x_*$:
		\begin{equation*}
			\frac{d}{dx} \Big| _{x=x_*} (e^{-\rho x} - 1 + B x) = 0 \rightarrow -\rho e^{-\rho x_*} + B = 0 \rightarrow x_* = -\frac{1}{\rho} \ln \frac{B}{\rho}.
		\end{equation*}
		Отсюда получаем уравнение, связывающее $\rho$ и $B$:
		\begin{equation*}
			2 - \frac{B}{\rho}\left(1 - \ln \frac{B}{\rho}\right) = e^{- \rho t_0} + B t_0.
		\end{equation*}
	\end{proof}
	\textit{Замечание.} Сделаем замены $\rho x = B, t_0 B = y$, тогда уравнение примет вид:
	\begin{equation*}
		2 - x \left(1 - \ln x\right) = e^{- \frac{y}{x}} + y.
	\end{equation*}
	После замен $\rho x = B, t_0 \rho = y$ уравнение примет вид:
	\begin{equation*}
		2 - x \left(1 - \ln x\right) = e^{- y} + x y.
	\end{equation*}
	Такие представления уравнений могут быть использованы для исследования связи $\rho$ и $B$. Например, в случае когда $B$ велико и,
	как следствие, не выполнено условие теоремы \ref{lilreg}, в некоторых случаях благодаря предположениям на $y$ мы можем получить оценку
	$x$. Если $B / x$ велико, то можно утверждать, что $\rho$ велико по модулю. В этом случае, стратегия оптимального исполнения вырождается в
	TWAP.


	% Наконец, мы получили искомое уравнение
	% \begin{equation*}
	%         \frac{\Delta A_{k+1}}{\Delta t_{k+1}} - \frac{\Delta A_{k}}{\Delta t_{k}} = 
	%         -\rho \Delta A_k + \rho (\lambda + \kappa) x_{t_k} - \rho \kappa x_{t_{k+1}} + (\lambda + \kappa) \left(\frac{x_{t_{k+1}}}{\Delta t_{k+1}} - \frac{x_{t_k}}{\Delta t_{k}}\right).
	% \end{equation*}


	\section{Время между сделками} \label{timedistr}
	Здесь, на рисунках \ref{appstart}--\ref{append}, представлены распределения времён между сделками для всех исследуемых активов.

	\subimport{./fig/}{figinc.tex}

	\section{Разные промежутки агрегации данных} \label{aggrnot1}

	\subimport{./tab/Appendix/}{Agr_CU.tex}
	\subimport{./tab/Appendix/}{Agr_SE.tex}




\end{appendices}   % Do not change this line