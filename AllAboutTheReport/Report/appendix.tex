\begin{appendices} % Do not change this line (if you have appendices). 
                   % Otherwise, completely delete the contents of this file



	\section{Обоснование метода}

        Из определения модели имеем три уравнения:
        \begin{equation} \label{rp1}
                A_{t_k} = V_{t_k} + \frac{s}{2} + \sum _{i=0} ^{k-1} x_{t_i} \kappa e^{- \rho (t_k - t_i)}
        \end{equation}
        \begin{equation}\label{rp2}
                V_{t_{k+1}} = V_{t_k} + \lambda x_{t_k} \rightarrow V_{t_{k+1}} - V_{t_k} = \lambda x_{t_{k}}
        \end{equation}
        \begin{equation} \label{rp3}
                D_{t_k} = A_{t_k} - V_{t_k} - \frac{s}{2}
        \end{equation}
        Следующее замечание является основополагающим в нашей методологии подбора параметра $\rho$.
        Здесь и далее $\Delta t_{k+1} := t_{k+1} - t_k, \Delta A_{k+1} := A_{k+1} - A_k$.
        \begin{lemma} \label{mainregrOW}
                В модели Обижаевой-Ванга:
                \begin{equation*}
                        \Delta A_k = D_{t_k} (e^{- \rho \Delta t_k} - 1) + x_{t_k} \kappa e^{- \rho \Delta t_k} + \lambda x_{t_k} .
                \end{equation*}
        \end{lemma}
        \begin{proof}
                Сперва покажем, что 
                \begin{equation*} \label{DeltaDk}
                    \Delta D_{k} = D_{t_k} (e^{- \rho \Delta t_k} - 1) + x_{t_k} \kappa e^{- \rho \Delta t_k}.
                \end{equation*}
                Пользуясь \eqref{rp1} и \eqref{rp3}, получаем
                \begin{align*}
                        D_{t_k} &= \sum _{i=0} ^{k-1} x_{t_i} \kappa e^{- \rho (t_k - t_i)} \\
                        \Delta D_{k} &= \sum _{i=0} ^k x_{t_i} \kappa e^{- \rho (t_{k+1} - t_i)} 
                        - \sum _{i=0} ^{k - 1} x_{t_i} \kappa e^{- \rho (t_k - t_i)}
                        = \sum _{i=0} ^{k - 1} x_{t_i} \kappa (e^{- \rho (t_{k+1} - t_i)} - e^{- \rho (t_k - t_i)})
                        + x_{t_k} \kappa e^{- \rho (t_{k+1} - t_k)} = \\
                        &= \sum _{i=0} ^{k - 1} x_{t_i} \kappa e^{- \rho (t_k - t_i)} (e^{- \rho (t_{k+1} - t_k)} - 1)
                        + x_{t_k} \kappa e^{- \rho (t_{k+1} - t_k)} = D_{t_k} (e^{- \rho \Delta t_k} - 1) + x_{t_k} \kappa e^{- \rho \Delta t_k}.
                \end{align*}
                Теперь покажем, что 
                \begin{equation*}
                        \Delta A_k = D_{t_k} (e^{- \rho \Delta t_k} - 1) + x_{t_k} \kappa e^{- \rho \Delta t_k} + \lambda x_{t_k} .
                \end{equation*}
                Из \eqref{rp2} и \eqref{rp3} имеем
                \begin{equation*}
                        \Delta D_k = D_{t_{k+1}} - D_{t_k} = A_{t_{k+1}} + V_{t_{k+1}} - A_{t_k} - V_{t_k} = \Delta A_k - \Delta V_k.
                \end{equation*}
                Отсюда имеем, что 
                \begin{equation*}
                        \Delta A_k = \Delta D_k + \Delta V_k .
                \end{equation*}
                Подставив сюда \eqref{DeltaDk}, получаем утверждение леммы. 
        \end{proof}

        Мы считаем, что при всей своей простоте оно очень ценно, поскольку даёт путь к выводу регрессионного уравнения.

        \begin{theorem} \label{GenIntT}
                Интерполируя экспоненту в выражении 
                \begin{equation*} \label{DeltaDk}
                        \Delta D_{k} = D_{t_k} (e^{- \rho \Delta t_k} - 1) + x_{t_k} \kappa e^{- \rho \Delta t_k}.
                \end{equation*}
                функцией вида $e^{- \rho \Delta t_k} = 1 - B \rho \Delta t_k + E(\rho \Delta t_k) $, где $E(\rho \Delta t_k)$ ---
                ошибка интерполяции,
                можно получить регрессионное уравнение вида:
                \begin{align*}
                        & \frac{\Delta A_{k+1}}{\Delta t_{k+1}} - \frac{\Delta A_{k}}{\Delta t_{k}} = 
                        -B \Delta A_k + B (\lambda + \kappa) x_{t_k} - B \kappa x_{t_{k+1}} 
                        + (\lambda + \kappa) \left(\frac{x_{t_{k+1}}}{\Delta t_{k+1}} - \frac{x_{t_k}}{\Delta t_{k}}\right) + E (\rho \Delta t_{k+1}) - E (\rho \Delta t_k).
                \end{align*}  
        \end{theorem}
        \begin{proof}
                Пусть $f(\rho \Delta t_k)$ --- некоторая функция, приближающая экспоненту с ошибкой $(\rho \Delta t_k)$.
                Тогда, подставив эту функцию в уравнение \eqref{mainregrOW}, получаем
                \begin{equation*}
                        \frac{\Delta A_k}{\Delta t_k} = D_{t_k} (\frac{f(\rho \Delta t_k) - 1}{\Delta t_k}) 
                        + x_{t_k} \kappa {\frac{f(\rho \Delta t_k)}{\Delta t_k}} + \lambda \frac{x_{t_k}}{\Delta t_k} . 
                \end{equation*}
                Потребуем, чтобы при рассмотрении разности $\frac{\Delta A_{k+1}}{\Delta t_{k+1}} - \frac{\Delta A_k}{\Delta t_k}$ 
                выполнялось условие 
                $\frac{f(\rho \Delta t_{k+1}) - 1}{\Delta t_{k+1}} = \frac{f(\rho \Delta t_k) - 1}{\Delta t_k}$. В этом случае, 
                рассматривая разность делённых на время асков, можно
                исключить из уравнения ненаблюдаемый временной ряд $D_{t_k}$:
                \begin{align*}
                        & R := (x_{t_{k+1}} \kappa + D_{t_{k+1}}) \frac{E(\rho \Delta t_{k+1})}{\Delta t_{k+1}} -
                         (x_{t_k} \kappa + D_{t_k}) \frac{E(\rho \Delta t_k)}{\Delta t_k} \\
                        \frac{\Delta A_{k+1}}{\Delta t_{k+1}} - \frac{\Delta A_{k}}{\Delta t_{k}} &=
                        - B D_{t_{k+1}} + x_{t_{k+1}} \kappa \left(\frac{1}{\Delta t_{k+1}} - \rho \right) + \lambda \frac{x_{t_{k+1}}}{\Delta t_{k+1}}
                        + B D_{t_{k}}   - x_{t_{k}}   \kappa \left(\frac{1}{\Delta t_{k}} - \rho \right)   - \lambda \frac{x_{t_k}}    {\Delta t_{k}} 
                        + R = \\
                        &= -B (\Delta A_k - \Delta V_k) + (\lambda + \kappa) \left(\frac{x_{t_{k+1}}}{\Delta t_{k+1}} - \frac{x_{t_k}}{\Delta t_{k}}\right) 
                        - B \kappa (x_{t_{k+1}} - x_{t_{k}}) + R = \\
                        &= -B \Delta A_k + B (\lambda + \kappa) x_{t_k} - B \kappa x_{t_{k+1}} 
                        + (\lambda + \kappa) \left(\frac{x_{t_{k+1}}}{\Delta t_{k+1}} - \frac{x_{t_k}}{\Delta t_{k}}\right) + R.
                \end{align*} 

        \end{proof}
        \par
        Помимо всего прочего, в доказательстве показано, что такой подход к выводу регрессионного уравнения 
        не работает ни для какого более широкого класса непрерывных или монотонных функций, чем двучлены вида 
        $1 - B \rho \Delta t_k$,
        поскольку единственной непрерывной или монотонной функцией одной переменной, обладающей свойством
        однородности первой степени, необходимым для осуществления рассуждения выше, является функция 
        $g(\rho \Delta t_k) = f(\rho \Delta t_k) - 1 = - B \rho \Delta t_k $. 
        \par
        Впрочем, это не исключает возможности того, что существует иной путь вывода регрессионого уравнения,
        позволяющий рассмотреть более точную интерполяцию. Это интересный вопрос для отдельного исследования.  
        
        \begin{theorem}\label{lilreg}
                В регрессии                                                                                                                                                                                                                                                                                                                                                                                       
                \begin{equation*}
                        \frac{\Delta A_{k+1}}{\Delta t_{k+1}} - \frac{\Delta A_{k}}{\Delta t_{k}}
                        = -B \Delta A_k + B (\lambda + \kappa) x_{t_k} - B \kappa x_{t_{k+1}} + 
                        (\lambda + \kappa) \left(\frac{x_{t_{k+1}}}{\Delta t_{k+1}} - \frac{x_{t_k}}{\Delta t_{k}}\right),
                \end{equation*}
                где $x_{k}$ и $A_{k}$ --- глубина ордера и аск в момент времени $t_k$, соответственно, \\
                $\rho = B + O(\rho^2 \Delta t)$ .
        \end{theorem}
        \begin{proof}
        Разложим экспоненту в ряд Тейлора: 
        \begin{equation*}
                e^{- \rho \Delta t_k} = 1 - \rho \Delta t_k + (\rho \Delta t_k) ^2 \sum_{i=0}^{\infty} \frac{(\rho \Delta t_k)^i}{(i+2)!},
        \end{equation*}
        тогда из \ref{GenIntT}, получим:
        \begin{align*}
                & R := (x_{t_{k+1}} \kappa + D_{t_{k+1}}) \rho^2 \Delta t_{k+1}  \sum_{i=0}^{\infty} \frac{(\rho \Delta t_{k+1})^i}{(i+2)!} 
                - (x_{t_k} \kappa + D_{t_k}) \rho^2 \Delta t_k \sum_{i=0}^{\infty} \frac{(\rho \Delta t_k)^i}{(i+2)!} \\
                & \frac{\Delta A_{k+1}}{\Delta t_{k+1}} - \frac{\Delta A_{k}}{\Delta t_{k}} = -\rho \Delta A_k + \rho (\lambda + \kappa) x_{t_k} - \rho \kappa x_{t_{k+1}} 
                + (\lambda + \kappa) \left(\frac{x_{t_{k+1}}}{\Delta t_{k+1}} - \frac{x_{t_k}}{\Delta t_{k}}\right) + R.
        \end{align*} 
        \end{proof}
        % Полагая, что $e^{- \rho \Delta t_k} = 1 - \rho \Delta t_k + \frac{1}{2} (\rho \Delta t_k)^2 + o((\rho \Delta t_k) ^3)$, имеем:
        % \begin{align*}
        %         \Delta D_{t_k} &=  D_{t_k} (e^{- \rho \Delta t_k} - 1) + x_{t_k} \kappa e^{- \rho \Delta t_k} 
        %         =  - \rho D_{t_k} \Delta t_k + x_{t_k} \kappa (1 - \rho \Delta t_k) + (x_{t_k} \kappa + D_{t_k}) o((\rho \Delta t_k) ^3) 
        % \end{align*}

        \section{Что делать, если $\rho$ получается большим?} \label{AppendixBigRho}

        Теорема \ref{lilreg} даёт указание к действию, когда $\rho^2 \Delta t$ мал. Но что делать, если это условие систематически 
        нарушается, например, актив настолько ликвиден, что $\rho$ существенно превосходит единицу? \par
        % Если предположить, что данные действительно подчинены экспоненциальному закону, то, в такой постановке, метод наименьших 
        % квадратов фактически будет решать задачу:
        % \begin{equation*}
        %         \sum _{i} \Delta t_i \left(e^{- \rho \Delta t_i} - 1 + B \Delta t_i \right) \rightarrow \min.
        % \end{equation*} 
        % Её решение легко найти аналитически:
        % \begin{equation*}
        %         B = \frac{\sum _{i} \Delta t_i}{\sum _{i} \Delta t_i^2} - \frac{\sum _{i} \Delta t_i e^{-\rho \Delta t_i}}{\sum _{i} \Delta t_i^2} 
        %         = \frac{\sum _{i} \Delta t_i ( 1 - e^{-\rho \Delta t_i})}{\sum _{i} \Delta t_i^2}.
        % \end{equation*} 
        % При $\rho \Delta t_i \rightarrow 0$ имеем $1 - e^{-\rho \Delta t_i} \rightarrow \rho \Delta t_i$, 
        % а значит, $B \rightarrow \rho$, то есть такой подход подтверждает и расширяет полученный ранее вывод. 
        % Однако, эта формула очень неудобна для численного вычисления $\rho(B)$ в иных случаях. Поэтому будем считать,
        % что при вычисленнии коэффициентов решается задача
        
        \begin{theorem}
                Если считать, что при большом $\rho \Delta t$ регрессия решает задачу
                \begin{equation*}
                        \min _{B \in \mathbb{R}} \max _{x \in [0, t_0]} |e^{- \rho x} - 1 + B x|,
                \end{equation*} 
                где $t_0$ некоторое "среднее" время между двумя соседними ордерами, то $B$ и $\rho$ связаны уравнением:
                \begin{equation*}
                        2 - \frac{B}{\rho}\left(1 - \ln \frac{B}{\rho}\right) = e^{- \rho t_0} + B t_0.
                \end{equation*} 
        \end{theorem}
        \begin{proof}
        Очевидно, разность под модулем обращается в ноль в двух точках ($0$ и $x_0$), если только прямая не является касательной к экспоненте.
        При этом, функция выпукла в промежутке $[0, x_0]$, а значит имеет там единственную точку экстремума. Из свойств функции ясно, что $B$ 
        является решением задачи в том и только в том случае, когда:
        \begin{equation*}
                - extr \{e^{-\rho x} - 1 + B x \}_{x \in [0, x_0]} = e^{-\rho t_0} - 1 + B t_0.
        \end{equation*} 
        Легко найти точку экстремума $x_*$:
        \begin{equation*}
                \frac{d}{dx} \Big| _{x=x_*} (e^{-\rho x} - 1 + B x) = 0 \rightarrow -\rho e^{-\rho x_*} + B = 0 \rightarrow x_* = -\frac{1}{\rho} \ln \frac{B}{\rho}.
        \end{equation*} 
        Отсюда получаем уравнение, связывающее $\rho$ и $B$:
        \begin{equation*}
                2 - \frac{B}{\rho}\left(1 - \ln \frac{B}{\rho}\right) = e^{- \rho t_0} + B t_0.
        \end{equation*} 
        \end{proof}
        \textit{Замечание.} Сделаем замены $\rho x = B, t_0 B = y$, тогда уравнение примет вид:
        \begin{equation*}
                2 - x \left(1 - \ln x\right) = e^{- \frac{y}{x}} + y.
        \end{equation*} 
        После замен $\rho x = B, t_0 \rho = y$ уравнение примет вид:
        \begin{equation*}
                2 - x \left(1 - \ln x\right) = e^{- y} + x y.
        \end{equation*} 
        Такие представления уравнений могут быть использованы для исследования связи $\rho$ и $B$. Например, в случае когда $B$ велико и,
        как следствие, не выполнено условие теоремы \ref{lilreg}, в некоторых случаях благодаря предположениям на $y$ мы можем получить оценку
        $x$. Если $B / x$ велико, то можно утверждать, что $\rho$ велико по модулю. В этом случае, стратегия оптимального исполнения вырождается в
        TWAP.


        % Наконец, мы получили искомое уравнение
        % \begin{equation*}
        %         \frac{\Delta A_{k+1}}{\Delta t_{k+1}} - \frac{\Delta A_{k}}{\Delta t_{k}} = 
        %         -\rho \Delta A_k + \rho (\lambda + \kappa) x_{t_k} - \rho \kappa x_{t_{k+1}} + (\lambda + \kappa) \left(\frac{x_{t_{k+1}}}{\Delta t_{k+1}} - \frac{x_{t_k}}{\Delta t_{k}}\right).
        % \end{equation*}
	

        \section{Время между сделками} \label{timedistr}
        Здесь, на рисунках \ref{appstart}--\ref{append}, представлены распределения времён между сделками для всех исследуемых активов.

        \subimport{./fig/}{figinc.tex}

        \section{Разные промежутки агрегации данных} \label{aggrnot1}

        \subimport{./tab/Appendix/}{Agr_CU.tex}
        \subimport{./tab/Appendix/}{Agr_SE.tex}




\end{appendices}   % Do not change this line