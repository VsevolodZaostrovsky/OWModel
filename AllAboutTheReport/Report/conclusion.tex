\conclusion % Do not change this line


На наш взгляд, предложенная методика обладает очень важным для индустрии достоинством: 
она очень проста как теоретически, так и вычислительно. Более того, она настолько проста теоретически,
что кажется невозможным, чтобы что-то пошло не так.
\par
В первом приближении, она очень неплохо себя зарекомендовала, результаты оценки параметров согласуются
как с метриками ликвидности, так и с экономической интуицией. Порядок параметров согласуется с теоретическими
предположениями Обижаевой и Ванга.
\par
К тому же, в данных было обнаружено несколько примечательных закономерностей.
Во-первых, данных очень распространены спайки. Это вполне резонно, поскольку на российском рынке к бирже имеет доступ
очень большое количество физических лиц, которые не располагают огромными средствами (по сравнению с институционалами)
и нуждаются в проведении небольших сделок.
\par
Во-вторых, создается ощущение, будто бы даже при исполнении большого ордера, институциональный игрок старается не сбивать
больше одного ценового уровня, как будто торгуя лимитными ордерами с лимитом, равным аску.
\par
В-третьих, на русском рынке довольно много активов с чрезвычайно низкой интенсивностью торгов. Их изучение, как будто,
стоит производить средствами макроэкономики и теории игр, а не статистики и численных методов.
\par
В-четвертых, наблюдается любопытная закономерность связанная со временем между сделками. К тому же, похоже, что
это время распределено экспоненциально.
\par
В данном случае обычная "виртуальная" методика бэк-тестирования категорически не подходит. Полученные в следствие её результаты
не будут иметь никакого смысла, поскольку невозможно учесть одно из ключевых предположений модели Обижаевой--Ванга:
большой ордер определенным образом изменяет характер восстановления стакана. Для полноценного тестирования необходим 
очень хорошо настроенный симулятор рынка. Его создание --- это темя для отдельного исследования.