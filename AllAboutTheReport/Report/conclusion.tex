\conclusion % Do not change this line


На наш взгляд, предложенная методика обладает очень важным для индустрии достоинством: 
она очень проста как теоретически, так и вычислительно. Более того, она настолько проста теоретически,
что кажется невозможным, чтобы что-то пошло не так. Тем не менее, она предполагает существенное требование ---
достаточно высокую частоту торгов, что выполнено не для всех активов. В ином случае, корректность её работы
ничем не гарантируется. 
\par
Мы считаем, что модель Обижаевой--Ванга неприменима для российских рынков, поскольку активы в нём как будто
подразделены на две категории:

\begin{enumerate}
    \item Активы с настолько низкой интенсивностью торгов, что для них, пожалуй, бессмысленна любая
    стратегия оптимального исполнения более сложная, чем TWAP или VWAP.
    \item Активы с относительно высокой интенсивностью торгов (хотя несопоставимой с европейскими рынками).
    Они, в силу особенностей проведения торгов, в частности, отсутствия больших ордеров, двигающих рынок 
    и наличия скачков аска, в парадигме модели являются активами с очень большой упругостью, хотя не похоже,
    что они являются таковыми de facto.
\end{enumerate}
Из-за этого, мы фактически не смогли полноценно протестировать методику на российских данных. Тем не менее, 
мы верим, что она сработает для рынков с более высокой интенсивностью торгов. \par
Эти же особенности, вероятно, делают невалидной любую модель с существенными теоретическими предпосылками.
Таким образом, на наш взгляд, оптимальным рецептом для индустрии является исследование алгоритмов не накладывающих
обременительных предпосылок, таких как, к примеру, VWAP и TWAP.