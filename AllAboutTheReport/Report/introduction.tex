\introduction % Do not change this line

Модель Обыжаевой--Ванга - это модель торговли на финансовом рынке, которая широко используется в количественных финансах.
Разработанная Анной Обыжаевой и Цзян Вангом в 2013 году, модель используется для анализа динамики финансовых
рынков и принятия торговых решений. Модель привлекла значительное внимание в финансовой индустрии
благодаря своему вниманию к свойству упругости (resiliency) стакана, описывающему важный эмпирический факт: спрос/
предложение финансовых ценных бумаг, как правило, не является идеально эластичным. Упругость --- скорость, с которой спрос/предложение
восстанавливается до устойчивого состояния после совершения сделки --- характеризует начало нового этапа в разработке
моделей оптимального исполнения. В нашем исследовании мы разрабатываем практический способ использования этой теории.
\par
Этот факт справедлив даже для ликвидных европейских рынков, если же мы говорим о гораздо менее ликвидных российских рынках,
пренебрежение им может привести к катастрофическим последствиям. Основное отличие модели Обыжаевой--Ванга  от других как раз
в том, что в ней ключевую роль играет устойчивость.
\par
Мы исследуем задачу оптимального исполнения. Иными словами, если кто-то хочет продать или купить определенное количество актива
достаточно большое, чтобы оказать существенное влияние на рынок, что он должен делать? Чтобы сформулировать задачу на языке
математики, нам нужно будет изучить основные принципы биржевой структуры.


