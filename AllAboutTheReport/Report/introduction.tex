\introduction % Do not change this line

The Obyzhaeva—Wang model is a financial market trading model that is widely used in quantitative finance. 
Developed by Anna Obyzhaeva and Jiang Wang in 2013, the model is used to analyze the dynamics of financial 
markets and make trading decisions. The model has gained significant attention in the financial industry 
for its attention to the resiliency, that describes an important empirical fact: the supply/demand of 
financial securities is in general not perfectly elastic. Resiliency --- the speed at which supply/demand 
recovers to its steady state after a trade --- characterizes the beginning of a new stage in the development 
of optimal execution models. In our research we develope a practical way to utilize that object. 
\par
This fact is true even for liquid European markets, if we talk about much less liquid Russian markets, 
neglecting this fact can be disastrous. The main difference between OW model and others is precisely 
that resiliency plays a key role in it. 
\par
Our point of interest here is an optimal execution problem. If one wants to sell or buy an amount of an asset 
large enough to have a significant impact on the market, what should he do? To write the problem in the language 
of mathematics, we will need to study the basic principles of the exchange structure. 


