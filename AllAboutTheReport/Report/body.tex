% Write the report here. You can use the nested imports via 'subimport', see the following link:
% https://www.overleaf.com/learn/latex/Management_in_a_large_project#Using_the_import_package
% You also can see an example at line n.16.

\section{Основные понятия}
Определим несколько понятий, которые понадобятся нам в дальнейшем.
\begin{definition}
    \textbf{Бид} (bid) --- это самая высокая цена, которую покупатель готов заплатить за актив. 
    Далее, бид в момент времени $t$ будет обозначаться $B_t$.
    \textbf{Аск} (ask) --- это самая низкая цена, за которую продавец готов предоставить актив. 
    Далее, аск в момент времени $t$ будет обозначаться $B_t$.
    \textbf{Бид-аск спрэд} (bid–ask spread) $s$: $s = A_t - B_t$.
    \textbf{Мид} (mid-quote price): $V_{t} = \frac{A_{t} + B_{t}}{2}$.
\end{definition}
Начнем с рассмотрения структуры лимитной книги заявок (Limit Order Book -- LOB). 
В рамках этой парадигмы организации биржевых торгов, у каждого участника есть две возможности:
\begin{itemize}
    \item выразить желание купить или продать определённое количество единиц актива по определенной цене. 
    В этом случае, биржа запомнит пару цена--количество. Множество этих пар составляет 
    лимитную книгу заявок. На рисунке \ref{LOBpic} изображено традиционное представление лимитной
    книги заявок.
    \item выразить желание купить или продать определённое количество единиц актива немедленно. 
    В этом случае он немедленно получит запрошенное количество акций (если на бирже есть необходимое количество)
    по лучшей возможной цене: к примеру, в случае покупки, если на верхнем ценовом уровне не будет достаточного
    количества единиц актива для удовлетворения заявки, то будут взяты активы из следующего ценового уровня.
    Таким образом, не гарантируется, что итоговая цена одной единицы актива будет совпадать с аском.
    
\end{itemize}

\begin{figure}
    \includegraphics[scale=0.8]{fig/Graphical-representation-of-the-Limit-Order-Book.png}
    \caption{Графическое представление лимитной книги заявок}
    \label{LOBpic}
\end{figure}



Таким образом, если опустить некоторые подробности, существует два вида заявок:
\begin{definition}
    \textbf{Лимитная заявка (ордер)}(limit order) представляет собой распоряжение на покупку или продажу ценной бумаги по определенной цене или выше. 
    Этот тип заявки гарантирует цену исполнения, но не гарантирует само исполнение.
\end{definition}
\begin{definition}
    \textbf{Рыночная заявка (ордер)} (market order) представляет собой распоряжение на немедленную покупку или продажу ценной бумаги. 
    Этот тип заявки гарантирует, что она будет исполнена, но не гарантирует цену исполнения.
\end{definition}

% \begin{figure}
%     \includegraphics[scale=0.4]{fig/market-vs-limit-1024x614.png}
%     \caption{The difference between market order and limit order}
%     \label{fig:mvslim}
% \end{figure}

Теперь ясно, что если мы хотим продать или купить актив в количестве, достаточно большом, чтобы он мог оказать существенное
влияние на рынок, мы не должны делать это одной заявкой: это было бы очень дорого, поскольку крупный ордер
удалил бы все верхние уровни в лимитной книге заявок. Поэтому на практике все крупные заявки разбиваются на большое количество мелких.
Например, можно просто разделить ордер на N равных частей и продавать их через равные промежутки времени (это называется TWAP).
Но есть ли лучшее решение?


\section{Подход Обижаевой и Ванга к формализации проблемы}
В попытке найти лучшее решение, мы рассматриваем модель Обижаевой--Ванга,
в терминах которой задача имеет следующий вид: \par
\begin{align*} \label{oEproblem}
    J_0 &= \min _{\{x_0 \cdots x_N \}} E_0 \left[ \sum _{n=0}^N [A_{t_n} + x_n /(2q)] x_n\right],  \\
    A_{t_n} &= F_{t_n} + \lambda (X_0 - X_{t_n}) + s/2 + \sum _{i=0}^{n-1} x_i \kappa e^{- \rho \tau (n - i)},
 \end{align*}
 
где
\begin{itemize}
 \item трейдер должен купить $\mathbf{X_0}$ единиц актива за фиксированный период времени $[0,T]$;
 \item $x_{t_n}$ размер оредра в момент времени $t_n = \tau n$ (здесь, $\tau = T / N$); 
 \item $X_{t_n} := X_0 - \sum _{t_k < t_n} x_{t_k}$;
 \item $B_{t_n}$ и $A_{t_n}$ --- бид и аск в момент времени $t_n$; 
 \item $V_{t_n} = \frac{A_{t_n} + B_{t_n}}{2}$ --- мид; 
 \item $s$ --- бид-аск спрэд;
 \item $F_t$ --- фундаментальная (справедливая) цена актива;
 \item $q(P)$ распределение лимитной книги заявок $P$ (по ценам $[a, b]$ доступно $\int_a^b q(p) dP$ единиц актива);
 % \item $q$, $\lambda$ and $\rho$ is a LOB density, the permanent price impact and the resiliency.
 \item параметр $\lambda$ --- постоянный маркет импакт (справдливая цена $V_t$ в результате исполнения ордера объема $x$ меняется по закону: $V_{t+} = V_t + \lambda x$);
  \item $\kappa = \frac{1}{q} - \lambda $;
 \item параметр $\rho$ --- упргуость стакана (resiliency).
\end{itemize}

Данная задача была решена в статье \cite{obizhaeva2013optimal}:
\begin{theorem}
    Решение проблемы оптимального исполнения:
    \begin{multline*}
        x_n = - \frac{1}{2} \delta_{n + 1} [D_{t_n} (1 - \beta_{n + 1} e^{ - \rho \tau} + 2 \kappa \gamma_{n+1} e^{ - 2 \rho \tau}) 
         - X_{t_n} (\lambda + 2 \alpha_{n+1} - \beta_{n+1}\kappa e^{ - \rho \tau}) ], 
    \end{multline*}
    где $x_N = X_N$ и $D_t = A_t - V_t - s/2$. Ожидаемая цена будущих сделок в рамках
    стратегии оптимального исполнения меняется по закону
    \begin{equation*}
        J_{t_n} = (F_{t_n} + s/2) X_{t_n} + \lambda X_0 X_{t_n} + \alpha_n X_{t_n} ^2 + \beta_{n} D_{t_n} X_{t_n} + \gamma_n D_{t_n}^2, 
    \end{equation*}
    где коэффициенты $\alpha_{n+1}$, $\beta_{n+1}$, $\gamma_{n+1}$ и $\delta_{n+1}$ определяются рекурснивно по формулам:
    \begin{equation*}
        \alpha_{n} = \alpha_{n+1} - \frac{1}{4} \delta _{n+1} (\lambda + 2 \alpha_{n+1} - \beta_{n+1} \kappa e^{- \rho \tau})^2, 
    \end{equation*}
    \begin{multline*}
        \beta_{n} =  \beta_{n+1} e^{- \rho \tau} + \frac{1}{2} \delta _{n+1} (1 - \beta_{n+1} e^{- \rho \tau} 
         + 2 \kappa \gamma_{n+1} e^{- 2 \rho \tau}) (\lambda + 2 \alpha_{n+1} - \beta_{n+1} \kappa e^{-\rho \tau}), 
    \end{multline*}
    \begin{equation*}
         \gamma_n =   \gamma_{n+1} e^{- 2 \rho \tau} - \frac{1}{4} \delta _{n+1} (1 - \beta _{n+1} e^{- \rho \tau} 
    + 2 \gamma _{n+1} \kappa e^{- 2 \rho \tau})^2, 
    \end{equation*}
    где $\delta_{n+1} = [1/(2q) + \alpha_{n+1} - \beta_{n+1} \kappa e^{-\rho \tau} + \gamma _{n+1} \kappa ^2 e^{- 2 \rho \tau}]^{-1}$ и начальные условия
    \begin{equation*}
        \alpha_{N} = 1/(2q) - \lambda, \;\;\;\;\;\;\; \beta_N = 1, \;\;\;\;\;\;\; \gamma_N = 0.
    \end{equation*}
\end{theorem}

В нашем исследовании мы будем рассматривать предел этого решения.

\begin{theorem}
    При $N \rightarrow \infty$, стратегия оптимального исполнения принимает вид:
    \begin{align*}
        & \lim _{N \rightarrow \infty} x_0 = x_{t = 0} = \frac{X_0}{\rho T + 2}, \\
        & \lim _{N \rightarrow \infty} x_n / (T/N) = \dot X _t = \frac{\rho X_0}{\rho T + 2}, \;\;\;\;\;\; t \in (0, T), \\
        & \lim _{N \rightarrow \infty} x_0 = x_{t = 0} = \lim _{N \rightarrow \infty} x_n / (T/N) = x_{t=T}=  \frac{X_0}{\rho T + 2}.  %\\
    \end{align*}
    где $x_0$ первая сделка за отведенный период, $x_N$ --- последняя, и $\dot X _t$ скорость трейдинга между ними.
\end{theorem}

Таким образом, мы имеем явную формулу для стратегии оптимального исполнения. Но нам необходимо каким-либо образом найти
параметр $\rho$ для того, чтобы применить её на практике. \par

Другое возможное приложение параметра $\rho$ даёт формула, определяющая динамику аска после исполнения ордера глубины $x$:
\begin{equation*}
        A_t = \overline p _t + \frac{s}{2} + x \kappa e^{- \rho t}.
\end{equation*}
Поскольку, к примеру, на московской бирже, минимальный шаг цены составляет, в основном, около $0.05\%$ 
от спотовой цены (см. \ref{price_step}), согласно модели стакан будет полностью восстанавливаться 
после исполнения "большого" ордера (для которого $x \kappa \approx 0.01 * A_t $) примерно за $t = \frac{10}{\rho}$.
\begin{figure}
    \includegraphics[scale=0.64]{fig/price_step.png}
    \caption{Правила определения расчетного шага цены (сайт московской биржи).}
    \label{price_step}
\end{figure}

\section{Как подобрать $\rho$?}
Предлагаемая нами методология основана на следующей теореме.
\begin{theorem}\label{lilreg}
    Для упрощения записей введём обозначения:
    \[
    \Delta t_{k} := t_{k} - t_{k-1}, \; \; \; \;   
    \Delta A_{k} := A_{k} - A_{k-1}. 
    \]
    В регрессии                                                                                                                                                                                                                                                                                                                                                                                       
    \begin{equation*}
            \frac{\Delta A_{k+1}}{\Delta t_{k+1}} - \frac{\Delta A_{k}}{\Delta t_{k}}
            = -B \Delta A_k + B (\lambda + \kappa) x_{t_k} - B \kappa x_{t_{k+1}} + 
            (\lambda + \kappa) \left(\frac{x_{t_{k+1}}}{\Delta t_{k+1}} - \frac{x_{t_k}}{\Delta t_{k}}\right),
    \end{equation*}
    где $x_{t_k}$ и $A_{t_k}$ --- глубина ордера и аск в момент времени $t_k$, соответственно, \\
    $\rho = B + O(\rho^2 \Delta t)$ .
\end{theorem}
Таким образом, мы прадлагаем следующий эмпирический путь подбора параметра $\rho$:
\begin{enumerate}
    \item Подготавливаем и очищаем данные, проверяем их на соответствие модели Обижаевой--Ванга.
    \item Оцениваем по данным регрессию
    \begin{equation*}
        \frac{\Delta A_{k+1}}{\Delta t_{k+1}} - \frac{\Delta A_{k}}{\Delta t_{k}}
            = -B \Delta A_k + B (\lambda + \kappa) x_{t_k} - B \kappa x_{t_{k+1}} + 
            (\lambda + \kappa) \left(\frac{x_{t_{k+1}}}{\Delta t_{k+1}} - \frac{x_{t_k}}{\Delta t_{k}}\right),
    \end{equation*} 
    где
    \begin{itemize}
        \item $\Delta A_{k}$ изменение аска в следствие исполнения заявки размера $x_k$ в момент времени $t_k$.
        \item $\Delta t_{k}$ время между $k$ и $k + 1$ заявками.
    \end{itemize}
    \item Если $\rho^2 \Delta t$ мало, то считаем, что $B \approx \rho$.
\end{enumerate}
Если последнее условие не выполнено, то можно попытаться применить подход, изложенный в приложении \ref{AppendixBigRho}.


\section{Данные}
\subsection{Источники и предобработка}
Мы работали с ордерлогами (\href{https://fs.moex.com/f/3198/specifikacija-formata-dannyh.pdf}{спецификация}). Было написано несколько 
\href{https://github.com/VsevolodZaostrovsky/OWModel/tree/main/New%20data/data%20preparing}{программ},
которые, в совокупности, распаршивали исходные записи в таблицу из четвёрок (время, аск до исполнения ордера, аск после исполнения ордера, глубина ордера). 
Следует обратить внимание на то, что заявки, поглощающие несколько уровней, представляются в данных в виде последовательности заявок, поэтому 
перед началом обработки их следует объединить: объем ордера есть сумма объемов, аск до исполнения --- аск до исполнения первого ордера,
аск после исполнения --- аск после исполнения последнего ордера. Фрагмент таблицы, на котором представлен фрагмент обработанных данных
и схематичное изображение изложенной выше операции, изображен на рисунке \ref{datacsv}.
 Рассматривались данные за 
03.03.2021 (середина недели, месяц максимально удаленный от праздников, 
год, удаленный от начала СВО и начала эпидемии COVID.)
\begin{figure}
    \includegraphics[scale=0.35]{fig/datscsv.jpg}
    \caption{Данные после первого этапа парсинга}
    \label{datacsv}
\end{figure}

\subsection{Выбор инструмента и начальный анализ данных} \label{InitAnal}
В целом, известно, что российский рынок характеризуется низким уровнем ликвидности и низкой интенсивностью торгов. 
Для подбора параметров и статистических исследовании
нам нужно относительно большое количество данных, по этой причине
для очень существенной части активов, торгуемых на московской бирже, 
исследование в духе нашего было бы невозможно: даже в категориях 
активов, считающихся ликвидными,
встречается немало инструментов \footnote{Даже среди валютных пар большая 
часть имеет очень низкую интенсивность торгов, например, CHFRUB\_TOM, HKDRUB\_TOM, 
JPYRUB\_TOM, KZTRUB\_TOM.}, по которым совершается порядка нескольких десятков сделок в день (и даже меньше). 
Поэтому мы решили рассмотреть две категории активов: самые ликвидные валютные пары
и несколько наиболее ликвидных акций (мы исключили из рассмотрения акции ВТБ, поскольку их сильное дробление создаёт 
численные проблемы).
\par
При этом, почти у всех активов подавляющее большинство сделок не поглащает ни одного уровня 
(те фактическая цена совершения сделки равна аску, см. таблицы \ref{tableanalCUnew} и \ref{tableanalSE}). 
\par


\subsection{Спайки}

Данные характеризуются очень большим числом количеством резких и коротких скачков аска. 
Пересечение множеств
сотни самых больших сделок и сотни сделок, сдвинувших аск сильнее всего, пусто. Таким образом, похоже, что ордеров,
которые существенно двигают аск за счёт своего размера на рынке мало, то есть цена двигается, в основнов, по фундаментальным причинам.
На наш взгляд, такая структура данных может послужить основанием для того, чтобы усомниться в валидности модели Обижавеой--Ванга 
для российских данных, однако и не дает возможности с уверенносью сказать, что модель не применима. 
\par
Откуда же берутся эти скачки и что они из себя представляют? В стакане в каждый момент времени поддерживается тонкий баланс, любой
маленький ордер, более выгодный, чем лучшая цена, будет почти мгновенно поглощён рынком. Конечно, если возникает необходимость в продаже
или покуке небольшого количетсва актива, этим разумно воспользоваться.

\begin{example}
    Пусть аск равен $10$, а бид --- $9$. Если мы хотим продать небольшое количество актива, то можно, например, разместись ордер
    на продажу актива по цене $9.1$, тогда пока ордер не будет исполнен, аск станет равным $9.1$, однако, скорее всего, лот очень
    быстро выкупят, поскольку он намного лучше аска и даже мида. Таким образом, мы продадим актив по цене $9.1$ вместо $9$, 
    расплатившись за это довольно несущественным риском.
\end{example}

Мы проанализировали несколько конкретных спайков в ручную и оказалось, что происходит именно то, что описано в примере. 
Ясно, что эти скачки, в целом, не характеризуют
динамику рискового актива, поэтому целесообразно исключить их из исследуемого датасета. \par

% \begin{figure}
%     \includegraphics[scale=0.41]{fig/Palki.pdf}
%     \caption{График цен сделок USD000UTSTOM в первый час торгового дня}
%     \label{askgraph}
% \end{figure}

% \begin{table}[h!]
%     \begin{center}
%         \begin{tabular}{|c|c|c|c|c|c|}
%             \hline
%         Инструмент        & Общее число сделок & Не повлияли на аск & Спайки, среди сдвинувших аск \\ \hline
%         USD000UTSTOM      & $41963$ & $95\%$ & $21\%$ \\ \hline
%         USD000000TOD      & $13391$ & $86\%$ & $31\%$ \\ \hline
%         EUR\_RUB\_\_TOM   & $13383$ & $93\%$ & $32\%$ \\ \hline
%         EUR\_RUB\_\_TOD   & $4134 $ & $58\%$ & $42\%$ \\ \hline
%         USD000TODTOM      & $1343 $ & $72\%$ & $89\%$ \\ \hline
%         EURUSD000TOM      & $915  $ & $91\%$ & $26\%$ \\ \hline
%         EUR000TODTOM      & $265  $ & $72\%$ & $89\%$ \\ \hline
%         GBPRUB\_TOM       & $234  $ & $64\%$ & $ 0\%$ \\ \hline
%         CNYRUB\_TOM       & $167  $ & $51\%$ & $51\%$ \\ \hline
%         \end{tabular}
%     \end{center}
%     \label{tableanalCU}
%     \caption{Анализ сделок по наиболее торгуемым валютным парам (03.02.2020).}
% \end{table} 

\begin{table}[h!]
    \begin{center}
        \begin{tabular}{|c|c|c|c|c|c|}
            \hline
        Инструмент        & Число сделок & Не повлияли на аск  %& Спайки\footnote{Мы считаем, что сделка спайк, если она сдвинула} 
        \\ \hline
        USD000UTSTOM    & $28361$ &    $95\%$ %& 69\% 
        \\ \hline
        USD000000TOD    & $9624$ &     $92\%$ %& 61\% 
        \\ \hline
        EUR\_RUB\_\_TOM & $4021$ &  $79\%$ %& 73\% 
        \\ \hline
        EUR\_RUB\_\_TOD & $2535$ &  $57\%$ %& 52\% 
        \\ \hline
        USD000TODTOM    & $546$ &      $55\%$ %& 95\% 
        \\ \hline
        EURUSD000TOM    & $409$ &      $93\%$ %& 82\% 
        \\ \hline
        GBPRUB\_TOM     & $220$ &       $60\%$ %& 73\% 
        \\ \hline
        EUR000TODTOM    & $168$ &      $89\%$ %& 100\%
          \\ \hline
        \end{tabular}
    \end{center}
    \label{tableanalCUnew}
    \caption{Анализ сделок по наиболее торгуемым валютным парам (03.03.2021).}
\end{table} 

\begin{table}[h!]
    \begin{center}
        \begin{tabular}{|c|c|c|c|c|c|}
            \hline
        Инструмент   & Число сделок & Не повлияли на аск %& Спайки\footnote{text} 
        \\ \hline
        SBER &  $41647$  & $ 96\% $ % &  $ 48\% $
        \\ \hline
        GAZP &  $21566$  & $ 86\% $ %  & $ 50\% $ 
        \\ \hline
        VTBR &  $17100$  & $ 99\% $ % &  $ 34\%$
         \\ \hline
        YNDX &  $14110$  & $ 68\% $ %  & $ 39\% $ 
        \\ \hline
        MGNT &  $10929$  & $ 64\% $ %  & $ 30\% $ 
        \\ \hline
        LKOH &  $9759 $ &  $ 75\% $ % &  $ 35\% $
        \\ \hline
        ROSN &  $8648 $ &  $ 74\% $ % &  $ 39\%$
         \\ \hline
        PLZL &  $7121 $ &  $ 56\% $ % &  $ 27\% $
        \\ \hline
        SNGSP & $ 6032$  & $ 60\% $ %  & $ 24\% $ 
        \\ \hline
        MTLR &  $5985 $ &  $ 46\% $ % &  $ 75\%$
        \\ \hline
        \end{tabular}
    \end{center}
    \label{tableanalSE}
    \caption{Анализ сделок по наиболее торгуемым акциям (03.03.2021).}
\end{table} 


% В паре ликвидных валют EURUSD000TOM происходит лишь порядка пятисот сделок в день. Тем не менее, мы решили исследовать и эту пару,
% поскольу данных (если не слишком сильно их фильтровать) достаточно для обучения модели, в то же время, характер торгов, очевидно,
% другой. \par
% Таким образом, мы решили остановить свой выбор на парах
% USD000UTSTOM и EURUSD000TOM.
\subsection{Время между сделками}

При изучении данных мы обнаружили любопытную закономерность: между сделками либо 
больше секунды, либо миллисекунды, причем это справедливо для всех активов
(См. \ref{TimeDistrUSD000000TOD}, \ref{TimeDistrGAZPWP}, \ref{TimeDistrSNGSPWP} и \ref{timedistr}: 
здесб всюду сверху изображено распределение всего набора данных, а снизу --- срез).
К тому же распределение времени между сделками очень похоже для всех активов и напоминает экспоненциальное.
\par 
Этот факт, вместе с особенностями данных, описанных в разделе \ref{InitAnal}, даёт основание 
объединить близкие сделки. Рассмотрение одного крупного ордера, вместо совокупности мелких и близких ордеров,
на первый взгляд, не оказывает критически разрушительного влияния на способность данных репрезентовать экономику процесса торговли. 
К тому же, в этом случае данные гораздо лучше соответсвуют предположениям модели и возникнет существенно меньше численных проблем. 
Поэтому мы решили рассмотреть датасет, где сделки, которые ближе друг к другу, чем 1 милисекунда, собраны в одну.

\begin{figure}
    \includegraphics[scale=0.35]{fig/timedistr/CU/USD000000TOD.pdf}
    \caption{Распределение времени между сделками для USD000000TOD}
    \label{TimeDistrUSD000000TOD}
\end{figure}

\begin{figure}
    \includegraphics[scale=0.35]{fig/timedistr/SE/GAZP.pdf}
    \caption{Распределение времени между сделками для GAZP}
    \label{TimeDistrGAZPWP}
\end{figure}

\begin{figure}
    \includegraphics[scale=0.35]{fig/timedistr/SE/SNGSP.pdf}
    \caption{Распределение времени между сделками для SNGSP}
    \label{TimeDistrSNGSPWP}
\end{figure}


\section{Результаты регрессий}
    В терминах нашего датасета (см. вид данных на \ref{datacsv}), регрессия на $B$ выглядят следующим образом:
\begin{align*}
    \frac{\textrm{AskAfter}(k+1) - \textrm{AskBefore}(k+1)}{\textrm{Time}(k+2) - \textrm{Time}(k+1)} 
    - \frac{\textrm{AskAfter}(k) - \textrm{AskBefore}(k)}{\textrm{Time}(k+1) - \textrm{Time}(k)}
    = \\ 
    = -B (\textrm{AskAfter}(k) - \textrm{AskBefore}(k)) + B (\lambda + \kappa) \textrm{Volume}(k) 
    - B \kappa \textrm{Volume}(k+1) + \\
    + (\lambda + \kappa) \left(\frac{\textrm{Volume}(k+1)}{\textrm{Time}(k+2) - \textrm{Time}(k+1)} 
    - \frac{\textrm{Volume}(k)}{\textrm{Time}(k+1) - \textrm{Time}(k)}\right).
\end{align*}
\par
Для $B ^*$ всё аналогично, но меняется методология рассчёта аска. Теперь это не лучшая цена покупки
одной единицы актива, а лучшая цена покупки миллиона единиц актива. \par
\begin{table}[h!]
    \begin{center}
        \begin{tabular}{|c|c|c|c|c|c|}
            \hline
            Инструмент        & Число сделок & Объем & Объем (рубли) & $B$ & $B ^*$ \\ \hline
            % USD000UTSTOM    & $28 361$ & $2 939 534 000$ & $215 911 984 541$ & $0.4587  \pm 0.0053$ & $0.5993 \pm 0.0065 $ \\ \hline
            % USD000000TOD    & $ 9 624$ & $  575 390 000$ & $ 42 269 305 976$ & $0.1989  \pm 0.0054$ & $0.1609 \pm 0.004  $\\ \hline
            % EUR\_RUB\_\_TOM & $ 4 021$ & $  221 130 000$ & $ 19 674 466 474$ & $0.2327  \pm 0.0089$ & $0.1494 \pm 0.0065 $ \\ \hline
            % EUR\_RUB\_\_TOD & $ 2 535$ & $  119 543 000$ & $ 10 637 857 791$ & $0.1932  \pm 0.0058$ & $0.2267 \pm 0.0105 $ \\ \hline
            % CNYRUB\_TOM     & $   371$ & $   54 341 000$ & $    618 979 196$ & $0.0359  \pm 0.0075$ & $0.0109 \pm 0.0022 $ \\ \hline
            % GBPRUB\_TOM     & $   220$ & $    3 517 000$ & $    362 506 611$ & $0.0145  \pm 0.0183$ & $0.0023 \pm 0.0004 $ \\ \hline
            % CNY000000TOD    & $    55$ & $   26 814 000$ & $    305 090 159$ & $0.0562  \pm 0.0122$ & $0.1352 \pm 0.0261 $ \\ \hline
            % GBPRUB\_TOD     & $    38$ & $    1 543 000$ & $    158 511 630$ & $-0.0612 \pm 0.037 $ & $0.0817 \pm 0.0368 $ \\ \hline
    
            USD000UTSTOM    & $ 2.83 \cdot 10^4 $ & $ 2.94 \cdot 10^9 $ & $ 2.15 \cdot 10^{11} $ & $ 0.4587  \pm 0.0053 $ & $ 0.5993  \pm 0.0065 $ \\ \hline 
            USD000000TOD    & $ 9.62 \cdot 10^3 $ & $ 5.75 \cdot 10^8 $ & $ 4.22 \cdot 10^{10} $ & $ 0.1989  \pm 0.0054 $ & $ 0.1609  \pm 0.004  $ \\ \hline
            EUR\_RUB\_\_TOM & $ 4.02 \cdot 10^3 $ & $ 2.21 \cdot 10^8 $ & $ 1.96 \cdot 10^{10} $ & $ 0.2327  \pm 0.0089 $ & $ 0.1494  \pm 0.0065 $ \\ \hline 
            EUR\_RUB\_\_TOD & $ 2.53 \cdot 10^3 $ & $ 1.19 \cdot 10^8 $ & $ 1.06 \cdot 10^{10} $ & $ 0.1932  \pm 0.0058 $ & $ 0.2267  \pm 0.0105 $ \\ \hline 
            CNYRUB\_TOM     & $ 371             $ & $ 5.43 \cdot 10^7 $ & $ 6.19 \cdot 10^{8 } $ & $ 0.0359  \pm 0.0075 $ & $ 0.0109  \pm 0.0022 $ \\ \hline 
            GBPRUB\_TOM     & $ 220             $ & $ 3.51 \cdot 10^6 $ & $ 3.62 \cdot 10^{8 } $ & $ 0.0145  \pm 0.0183 $ & $ 0.0023  \pm 0.0004 $ \\ \hline 
            CNY000000TOD    & $ 55              $ & $ 2.68 \cdot 10^7 $ & $ 3.05 \cdot 10^{8 } $ & $ 0.0562  \pm 0.0122 $ & $ 0.1352  \pm 0.0261 $ \\ \hline 
            GBPRUB\_TOD     & $ 38              $ & $ 1.54 \cdot 10^6 $ & $ 1.58 \cdot 10^{8 } $ & $ -0.0612 \pm 0.037  $ & $ 0.0817  \pm 0.0368 $ \\ \hline
        \end{tabular}
    \end{center}
    \label{tableanal}
    \caption{$B$ и $B ^*$, вычисленные для разных валютных пар в агрегированном датасете.}
    \end{table} 
    Как видно, для валют абсолютно чётко прослеживается связь между разными метриками ликвидности и $B$,
    что даёт основания считать вычисленный коэффициент валидной мерой ликвидности актива. Кроме того, 
    во всех случаях $B$ довольно мал по модулю, так что можно считать, что $B = \rho$. Этого достаточно,
    чтобы практически осуществить оптимальное исполнение в соответствии с моделью Обижаевой--Ванга.
    Последние две валютных пары существенно выбиваются из общей колеи. Однако в их случае выборки ничтожно малы,
    так что, вероятно, методика просто напросто "переобучилась".\par
    \begin{table}[h!]
        \begin{center}
            \begin{tabular}{|c|c|c|c|c|c|}
                \hline
                Инструмент        & Число сделок & Объем & Объем (рубли) & $B$ & $B^*$ \\ \hline
                SBER & $4.16 \cdot 10^4$ & $5.66 \cdot 10^7$ & $1.58 \cdot 10^{10}$ & $ 0.7744 \pm 0.0063 $ & $ 0.9337 \pm 0.0072 $ \\ \hline
                GAZP & $2.15 \cdot 10^4$ & $2.85 \cdot 10^7$ & $6.38 \cdot 10^{ 9}$ & $ 0.8641 \pm 0.0086 $ & $ 0.7051 \pm 0.0082 $ \\ \hline
                YNDX & $1.41 \cdot 10^4$ & $6.91 \cdot 10^5$ & $3.50 \cdot 10^{ 9}$ & $ 0.773  \pm 0.0095 $ & $ 0.435  \pm 0.0072 $ \\ \hline
                MGNT & $1.09 \cdot 10^4$ & $4.30 \cdot 10^5$ & $2.18 \cdot 10^{ 9}$ & $ 0.7336 \pm 0.0113 $ & $ 0.4186 \pm 0.0095 $ \\ \hline
                LKOH & $9.75 \cdot 10^3$ & $7.79 \cdot 10^5$ & $4.43 \cdot 10^{ 9}$ & $ 0.4757 \pm 0.0078 $ & $ 0.433  \pm 0.009  $\\ \hline
                ROSN & $8.64 \cdot 10^3$ & $5.23 \cdot 10^6$ & $2.78 \cdot 10^{ 9}$ & $ 0.3116 \pm 0.0076 $ & $ 0.7418 \pm 0.012  $\\ \hline
                PLZL & $7.12 \cdot 10^3$ & $2.06 \cdot 10^5$ & $2.99 \cdot 10^{ 9}$ & $ 0.2234 \pm 0.0056 $ & $ 0.4871 \pm 0.0122 $ \\ \hline
                NGSP & $6.03 \cdot 10^3$ & $1.79 \cdot 10^7$ & $7.20 \cdot 10^{ 8}$ & $ 0.1842 \pm 0.0084 $ & $ 0.3528 \pm 0.0099 $ \\ \hline
                % MTLR & $5.98 \cdot 10^3$ & $1.21 \cdot 10^6$ & $8.60 \cdot 10^{ 7}$ & $ 0.5591 \pm 0.0092 $ & $ 0.5574 \pm 0.0112 $ \\ \hline
            \end{tabular}
        \end{center}
        \label{tableanal}
        \caption{$B$ и $B ^*$, вычисленные для разных акций в агрегированном датасете.}
        \end{table} 
    У акций также есть чёткая связь между разными метриками ликвидности и $B$,
    что даёт основания считать вычисленный коэффициент валидной мерой ликвидности актива. Кроме того, 
    во всех случаях $B$ довольно мал по модулю, так что можно считать, что $B = \rho$. Этого достаточно,
    чтобы практически осуществить оптимальное исполнение в соответствии с моделью Обижаевой--Ванга.
    \par
    Таким образом, наша методология, на первый взгляд, является состоятельной: инструменты выстраиваются в
    порядке, согласованном с традиционными метриками ликвидности. Кроме того, большая часть активов обладает
    достаточно малым $\rho$, чтобы условия основной теоремы можно было бы считать выполненными.


